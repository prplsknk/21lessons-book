\chapter*{Avant-propos}
\pdfbookmark{Foreword}{foreword}

Certains appellent ça une expérience mystique. D'autres appellent ça Bitcoin.

J'ai rencontré Gigi pour la première fois dans un de mes foyers spirituels --
Riga, Lettonie -- patrie de la conférence \textit{The Baltic Honeybadger}, où
les plus fervents fidèles de Bitcoin accomplissent un pèlerinage annuel. Après
une profonde conversation autour d'un déjeuner, le lien que Gigi et moi avions
tissé était aussi immuable qu'une transaction Bitcoin traitée quelques heures
plus tôt lorsque nous nous sommes salués.

Mon autre foyer spirituel, Christ Church à Oxford où j'ai eu le privilège
d'étudier pour mon MBA, est le lieu où m'est apparue la révélation
\enquote{terrier du lapin}. Comme Gigi, j'ai transcendé les sphères économiques,
techniques et sociales afin de laisser Bitcoin m'envelopper spirituellement.
Après avoir \enquote{acheté haut} pendant la bulle de novembre 2013, j'ai dû
tirer des enseignements très difficiles de l'interminable et destructeur marché
baissier de trois ans qui s'ensuivit. Ces 21 leçons m'auraient particulièrement
bien aidé à ce moment-là. La plupart sont simplement des vérités naturelles qui,
pour le néophyte, sont assombries par un film opaque et fragile. Cependant,
d'ici la fin de ce livre, cette façade volera en éclats.

Par une nuit très claire de la fin août 2016 à Oxford, quelques semaines
seulement après le piratage de la plateforme d'échange Bitfinex, j'ai fait une
halte contemplative au Master's Garden de Christ Church. C'était une période
compliquée et j'étais sur le point de craquer psychologiquement et
émotionnellement après ce qui m'a paru une éternité de torture. Pas pour les
pertes financières, non, mais bien à cause du vide spirituel écrasant que je
ressentais, isolé dans ma vision du monde. Si seulement un livre comme celui-ci
avait existé à l'époque, j'aurais pu me rendre compte que je n'étais pas seul.
Le Master’s Garden est un endroit particulier à mes yeux et aux yeux de beaucoup
avant moi au cours des siècles. C'est ici qu'un certain Charles Dodgson,
professeur de mathématiques à Christ Church, remarqua l'une de ses jeunes
élèves, Alice Liddell, fille du doyen. Dodgson, plus connu sous son nom de plume
Lewis Carroll, s'est inspiré d'Alice et du Master's Garden ; et par la magie de
ce vénérable gazon, j'ai plongé mon regard dans le crypto-abîme, qui me l'a
ardemment rendu, étouffant toute arrogance, giflant mon orgueil en plein visage.
J'étais enfin en paix.

21 Leçons vous embarque pour un véritable voyage vers Bitcoin, non seulement
philosophique, technologique et économique, mais aussi spirituel.

En se plongeant plus profondément dans la philosophie sobrement exposée dans 7
des 21 Leçons, avec assez de temps et de réflexion, il est possible d'aller
jusqu'à comprendre l'origine de toute chose. Ses 7 leçons sur l'économie rendent
compte, en des termes simples, de la façon dont nous sommes à la merci d'un
petit groupe de chapeliers fous et comment ils ont réussi à nous mettre des
œillères dans la tête, dans le cœur et à l'âme. Les 7 leçons sur la technologie
décrivent la beauté et la perfection technologiquement darwinienne de Bitcoin.
En tant que bitcoiner non technique, ces leçons apportent une étude pertinente
sur la nature fondamentalement technologique de Bitcoin et, de fait, sur la
nature de la technologie elle-même.

Dans cette expérience éphémère que nous appelons la vie, nous vivons, nous
aimons et nous apprenons. Mais qu'est-ce que la vie sinon une suite
chronologique d'événements ?

Parvenir au sommet de la montagne Bitcoin n'est pas chose aisée. C'est truffé de
faux sommets, le terrain est très accidenté et les crevasses sont omniprésentes,
prêtes à vous engloutir. Après la lecture de ce livre, vous comprendrez que Gigi
est le sherpa Bitcoin ultime. Je lui en serai toujours reconnaissant.

\begin{flushright}
  Hass McCook \\
  29 novembre 2019
\end{flushright}
