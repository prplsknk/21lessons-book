\chapter{La vie privée n'est pas morte}
\label{les:19}

\begin{chapquote}{Lewis Carroll, \textit{Alice in Wonderland}}
Les joueurs jouaient tous en même temps sans attendre leur tour ; ils se
disputaient sans arrêt et s’arrachaient les hérissons. Au bout d’un instant, la
Reine, entrant dans une furieuse colère, parcourut le terrain en tapant du pied
et en criant : \enquote{Qu’on lui coupe la tête ! Qu’on lui coupe la tête !} à
peu près une fois par minute.
\end{chapquote}

À en croire les experts, la vie privée est morte depuis les années
80\footnote{\url{https://bit.ly/privacy-is-dead}}. L'invention pseudonyme de
Bitcoin et d'autres événements de l'histoire récente montrent que ce n'est pas
vrai. La vie privée est vivante, bien qu'il ne soit pas facile d'échapper à cet
état de surveillance.

Satoshi a pris d'infinies précautions afin de se couvrir et de dissimuler son
identité. Dix ans plus tard, personne ne peut dire si Satoshi Nakamoto était une
personne ou un groupe, un homme, une femme ou un intelligence artificielle venue
du futur pour s'auto-amorcer afin de régner sur le monde. Théories complotistes
mises à part, Satoshi a choisi de s'identifier à un homme japonais, c'est
pourquoi je ne présume de rien en respectant son choix de genre et en le
désignant par \textit{il}.

\begin{figure}
  \includegraphics{assets/images/nope.png}
  \caption{Je ne suis pas Dorian Nakamoto.}
  \label{fig:nope}
\end{figure}

Quelle que soit sa vraie identité, Satoshi a réussi à la cacher. Il a créé un
précédent encourageant pour tous ceux qui désirent rester anonymes : c'est
possible d'avoir une vie privée en ligne.

\begin{quotation}\begin{samepage}
\enquote{Le chiffrement fonctionne. Les systèmes crypto robustes et correctement
implémentés sont l'une des rares choses sur lesquelles vous pouvez compter.}
\begin{flushright} -- Edward Snowden\footnote{Edward Snowden, réponses au
courrier des lecteurs \cite{snowden}}
\end{flushright}\end{samepage}\end{quotation}

Satoshi n'est pas le premier inventeur pseudonyme ou anonyme et ne sera pas le
dernier. Certains ont directement repris son style de publication pseudonyme,
comme Tom Elvis Jedusor, connu pour MimbleWimble~\cite{mimblewimble-origin},
quand d'autres ont publié des preuves mathématiques avancées en restant
totalement anonymes~\cite{4chan-math}.

C'est un nouveau monde étrange que nous habitons. Un monde où l'identité est
facultative, où les contributions sont acceptées sur la base du mérite et où les
gens peuvent collaborer et négocier librement. Il faudra quelques ajustements
pour être à l'aise avec ces nouveaux paradigmes, mais je crois fermement que
tout ceci a le potentiel pour rendre le monde meilleur.

Chacun d'entre nous devrait se souvenir que la vie privée est un droit humain
fondamental\footnote{Déclaration universelle des Droits de l'Homme,
\textit{Article 12}.~\cite{article12}}. Tant que le peuple exercera et défendra
ces droits, la bataille pour la vie privée sera loin d'être achevée.

\paragraph{Bitcoin m'a appris que la vie privée n'était pas morte.}

% ---
%
% #### Down the Rabbit Hole
%
% - [Universal Declaration of Human Rights][fundamental human right] by the United Nations
% - [A lower bound on the length of the shortest superpattern][anonymous] by Anonymous 4chan Poster, Robin Houston, Jay Pantone, and Vince Vatter
%
% [since the 80ies]: https://books.google.com/ngrams/graph?content=privacy+is+dead&year_start=1970&year_end=2019&corpus=15&smoothing=3&share=&direct_url=t1%3B%2Cprivacy%20is%20dead%3B%2Cc0
% [time-traveling AI]: https://blockchain24-7.com/is-crypto-creator-a-time-travelling-ai/
% ["I am not Dorian Nakamoto."]: http://p2pfoundation.ning.com/forum/topics/bitcoin-open-source?commentId=2003008%3AComment%3A52186
% [MimbleWimble]: https://github.com/mimblewimble/docs/wiki/MimbleWimble-Origin
% [anonymous]: https://oeis.org/A180632/a180632.pdf
% [fundamental human right]: http://www.un.org/en/universal-declaration-human-rights/
%
% <!-- Wikipedia -->
% [alice]: https://en.wikipedia.org/wiki/Alice%27s_Adventures_in_Wonderland
% [carroll]: https://en.wikipedia.org/wiki/Lewis_Carroll
