\chapter{L'Immaculée Conception}
\label{les:5}

\begin{chapquote}{Lewis Carroll, \textit{Alice au pays des merveilles}}
\enquote{Leur tête a disparu, [...]} répondirent les soldats.
\end{chapquote}

Ça parle à tout le monde lorsqu'une belle histoire donne lieu à une naissance.
Celle de Bitcoin est fascinante et ses détails sont plus importants qu'on
pourrait le croire à première vue. Qui est Satoshi Nakamoto ? Était-ce une seule
personne ou un groupe ? Était-ce un homme ou une femme ? Un extraterrestre qui
aurait voyagé dans le temps, ou une intelligence artificielle avancée ? Sans
tenir compte des théories absurdes, nous ne le saurons sans doute jamais. Et ça
a toute son importance.

Satoshi a choisi de rester anonyme. Il a fait germer la graine de Bitcoin. Il
est resté dans le coin suffisamment longtemps pour s'assurer que le réseau ne
connaîtrait pas une mort prématurée. Et il s'est évaporé.

Ce qui peut sembler une étrange pirouette à propos de l'anonymat est en réalité
une chose cruciale pour qu'un système soit vraiment décentralisé. Pas de
contrôle centralisé. Pas d'autorité centrale. Pas d'inventeur identifié.
Personne à poursuivre, à torturer, à faire chanter ou à extorquer. L'Immaculée
Conception d'une technologie.

\begin{quotation}\begin{samepage}
\enquote{L'une des meilleures choses que Satoshi ait faites a été de
disparaître.}
\begin{flushright} -- Jimmy Song\footnote{Jimmy Song, \textit{Pourquoi Bitcoin
est différent} \cite{bitcoin-different}}
\end{flushright}\end{samepage}\end{quotation}

\newpage

Depuis la naissance de Bitcoin, des milliers d'autres cryptomonnaies ont été
créées. Aucun de ces clones ne partage l'histoire de sa naissance. Si vous
voulez remplacer Bitcoin, vous allez devoir transcender cette histoire. Dans une
guerre d'idées, le récit impose la survie.

\begin{quotation}\begin{samepage}
\enquote{L'or a d'abord été travaillé sous forme de bijoux et utilisé pour le
troc il y a plus de 7000 ans. L'éclat captivant de l'or l'a mené à être
considéré comme un cadeau des dieux.}
\begin{flushright} Austrian Mint\footnote{The Austrian Mint, \textit{Gold: The
Extraordinary Metal} \cite{gold-gift-gods}}
\end{flushright}\end{samepage}\end{quotation}

Comme l'or il y a bien longtemps, Bitcoin pourrait être perçu comme un cadeau
des dieux. Mais contrairement à l'or, les origines de Bitcoin sont très
humaines. Et cette fois, nous connaissons les dieux du développement et de la
maintenance : des gens du monde entier, anonymes ou pas.

\paragraph{Bitcoin m'a appris que le narratif était important.}

% ---
%
% #### Down the Rabbit Hole
%
% - [Why Bitcoin is different][Jimmy Song] by Jimmy Song
% - [Gold: The Extraordinary Metal] by the Austrian Mint
%
% <!-- Down the Rabbit Hole -->
% [Jimmy Song]: https://medium.com/@jimmysong/why-bitcoin-is-different-e17b813fd947
% [Gold: The Extraordinary Metal]: https://www.muenzeoesterreich.at/eng/discover/for-investors/gold-the-extraordinary-metal
%
% <!-- Wikipedia -->
% [alice]: https://en.wikipedia.org/wiki/Alice%27s_Adventures_in_Wonderland
% [carroll]: https://en.wikipedia.org/wiki/Lewis_Carroll
