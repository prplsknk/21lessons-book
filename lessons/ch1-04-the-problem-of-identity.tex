\chapter{Le problème de l'identité}
\label{les:4}

\begin{chapquote}{Lewis Carroll, \textit{Alice au pays des merveilles}}
  \enquote{Qui es-tu ?} lui demanda-t-elle.
\end{chapquote}

Nic Carter, en hommage à Thomas Nagel qui posait cette même question à propos
des chauve-souris, a écrit un excellent article sur la question : quel effet
cela fait-il d'être un bitcoin ? Il y montre remarquablement qu'en général, les
blockchains publiques et ouvertes, et Bitcoin plus particulièrement, souffrent
du même dilemme que le bateau de Thésée\footnote{Dans la métaphysique de
l'identité, le bateau de Thésée est une expérience de pensée qui soulève la
question de savoir si un objet dont toutes les parties ont été remplacées reste
fondamentalement le même objet.~\cite{wiki:theseus}} : quel Bitcoin est le vrai
Bitcoin ?

\begin{quotation}\begin{samepage}
\enquote{Observez par exemple comment les éléments de Bitcoin font preuve de peu
de persistance. L'ensemble du code source a déjà été retravaillé, modifié et
étendu tant et si bien qu'il ne ressemble que difficilement à sa version
d'origine. [...] Les archives de qui possède quoi, le registre lui-même, est
virtuellement la seule caractéristique persistante du réseau [...]
Afin d'être vraiment perçu sans responsable, il faut tourner le dos à cette
solution simple qui consiste à ce qu'une entité puisse désigner une chaîne comme
étant la chaîne légitime.}
\begin{flushright} -- Nic Carter\footnote{Nic Carter, \textit{Quel effet cela
fait-il d'être un bitcoin ?} \cite{bitcoin-identity}}
\end{flushright}\end{samepage}\end{quotation}

Il semble que le progrès technologique nous force sans arrêt à prendre au
sérieux ces considérations philosophiques. Un jour ou l'autre, les voitures
autonomes feront face au dilemme du tramway de façon bien réelle, ce qui les
obligera à prendre des décisions d'ordre éthique sur quelles vies comptent et
quelles vies ne comptent pas.

Les cryptomonnaies, particulièrement depuis le premier hard-fork litigieux, nous
forcent à réfléchir et à se mettre d'accord sur la métaphysique de l'identité.
Il est intéressant de constater que les deux meilleurs exemples que nous avons
connu jusqu'à présent ont mené à deux réponses différentes. Le premier août
2017, Bitcoin se sépara en deux camps. Le marché décida que la chaîne inchangée
était le Bitcoin originel. Un an plus tôt, le 25 octobre 2016, Ethereum se
séparait en deux camps. Le marché décidait que la chaîne \textit{modifiée} était
l'Ethereum originel.

Aussi longtemps que ces réseaux de transfert de valeur existeront, pour peu
qu'ils soient correctement décentralisés, les questions posées par le
\textit{bateau de Thésée} devront sans cesse trouver des réponses.

\paragraph{Bitcoin m'a appris que la décentralisation entrait en contradiction
avec l'identité.}

% ---
%
% #### Down the Rabbit Hole
%
% - [What Is It Like to be a Bat?][in regards to a bat] by Thomas Nagel
% - [What is it like to be a bitcoin?] by Nic Carter
% - [Ship of Theseus], [trolley problem] on Wikipedia
%
% [in regards to a bat]: https://en.wikipedia.org/wiki/What_Is_it_Like_to_Be_a_Bat%3F
% [What is it like to be a bitcoin?]: https://medium.com/s/story/what-is-it-like-to-be-a-bitcoin-56109f3e6753
% [Ship of Theseus]: https://en.wikipedia.org/wiki/Ship_of_Theseus
% [trolley problem]: https://en.wikipedia.org/wiki/Trolley_problem
%
% <!-- Wikipedia -->
% [alice]: https://en.wikipedia.org/wiki/Alice%27s_Adventures_in_Wonderland
% [carroll]: https://en.wikipedia.org/wiki/Lewis_Carroll
