\chapter{L'argent}
\label{les:11}

\begin{chapquote}{Le Sage}
\enquote{Quand j'étais jeune, \ldots \\
j'ai bien entretenu la souplesse de mes muscles \\
en me frottant avec cette crème \\
-- un shilling la boite ! – \\
Voulez-vous m'en acheter un lot ?}
\end{chapquote}

Qu'est-ce que l'argent ? On l'utilise tous les jours, pourtant il est
étonnamment complexe de répondre à cette question. Nous en dépendons de toutes
les manières possibles et imaginables et si nous en manquons nos vies deviennent
très difficiles. Toutefois, nous réfléchissons rarement à cette chose qui fait
prétendument tourner le monde. Bitcoin m'a contraint à répondre inlassablement à
cette question : mais enfin c'est quoi, l'argent ?

Dans notre monde \enquote{moderne}, la plupart des gens penseront probablement à
des bouts de papier lorsqu'ils parleront d'argent, alors même qu'il n'est en
majorité qu'un nombre sur un compte bancaire. Notre argent est déjà fait de un
et de zéros, alors en quoi Bitcoin est-il différent ? Il l'est car en son sein,
c'est un \textit{type} de monnaie très différent de celle que l'on utilise
habituellement. Pour le comprendre, nous allons devoir nous pencher sur ce
qu'est la monnaie, comment elle a vu le jour et pourquoi l'or et l'argent ont
été utilisés durant la majeure partie de l'histoire du commerce.

\paragraph{}
Les coquillages, l'or, l'argent, le papier, le bitcoin. En fin de compte,
\textbf{la monnaie c'est ce dont les gens se servent}, peu importe son aspect et
sa forme, ou l'absence de celles-ci.

L'argent est ingénieux, en tant qu'invention. Un monde sans argent s'en
retrouverait compliqué à l'absurde : combien de poissons pour ces nouvelles
chaussures ? Combien de vaches pour acheter une maison ? Qu'est-ce que je fais
si je n'ai aucun besoin immédiat mais que je dois me débarrasser de mes pommes
bien mûres ? Pas besoin d'être très imaginatif pour comprendre qu'une économie
basée sur le troc serait terriblement inefficace.

Le truc excellent avec l'argent c'est qu'on peut l'échanger contre
\textit{n'importe quoi d'autre} -- c'est une sacrée invention ! Tel que Nick
Szabo\footnote{\url{http://unenumerated.blogspot.com/}} le résume
brillamment dans \textit{Shelling Out: The Origins of Money}
\cite{shelling-out}, les êtres humains ont utilisé toutes sortes de choses en
tant que monnaie : des perles de matériaux rares comme l'ivoire, des
coquillages, des os spécifiques, divers types de bijoux, puis plus tard des
métaux rares comme l'argent ou l'or.

\begin{quotation}\begin{samepage}
\enquote{En ce sens, il ressemble plus à un métal précieux. Au lieu que
ce soit l'offre qui change afin de maintenir la valeur, celle-ci est
prédéterminée et c'est la valeur qui change.}
\begin{flushright} -- Satoshi Nakamoto\footnote{Satoshi Nakamoto, dans une
réponse à Sepp Hasslberger \cite{satoshi-precious-metal}}
\end{flushright}\end{samepage}\end{quotation}

Tels les bons paresseux que nous sommes, nous ne passons pas trop de temps à
réfléchir à ce qui marche. L'argent, pour la plupart d'entre nous, ça fonctionne
très bien. Comme avec nos voitures ou nos ordinateurs, nous ne sommes obligés
d'y penser que lorsqu'un de ces trucs tombe en panne. Les personnes qui ont vu
leurs économies d'une vie s'évaporer avec l'hyperinflation savent très bien la
valeur d'une monnaie forte, tout comme ceux qui ont vu leurs amis et famille
disparaître à cause des atrocités de l'Allemagne nazie ou de l'Union Soviétique
connaissent très bien la valeur de la confidentialité.

Le problème avec l'argent, c'est qu'il est partout. Il représente la moitié de
chaque transaction, ce qui confère un pouvoir considérable à ceux qui sont
responsables de l'émission monétaire.

\begin{quotation}\begin{samepage}
\enquote{Étant donné que l'argent représente la moitié de chaque transaction
commerciale et que des civilisations entières s'épanouissent et s'effondrent
selon la qualité de leur monnaie, on parle ici d'un pouvoir incommensurable, un
pouvoir qui est passé sous silence. C'est le pouvoir de tisser des mirages qui
semblent vrais aussi longtemps qu'ils durent. C'est là le cœur du pouvoir de la
Réserve Fédérale.}
\begin{flushright} -- Ron Paul\footnote{Ron Paul, \textit{End the Fed}
\cite{end-the-fed}}
\end{flushright}\end{samepage}\end{quotation}

Bitcoin retire ce pouvoir pacifiquement, puisqu'il met fin à l'émission
monétaire sans recourir à la force.

L'argent a connu de multiples itérations. La plupart d'entre elles étaient
bonnes. Elles amélioraient notre monnaie d'une façon ou d'une autre. En
revanche, très récemment, ses rouages ont été corrompus. Aujourd'hui, la
quasi-totalité de notre argent est fabriqué \textit{de toutes pièces} par les
pouvoirs en place. Pour comprendre comment nous en sommes arrivés là, j'ai dû
étudier l'histoire de la monnaie et de son déclin consécutif.

Il reste encore à voir s'il faudra une série de catastrophes ou simplement un
effort éducatif monumental pour réparer cette corruption. Je prie les dieux de
la monnaie saine afin que ce soit le second.

\paragraph{Bitcoin m'a appris ce qu'était l'argent.}

% ---
%
% #### Down the Rabbit Hole
%
% - [End the Fed][Ron Paul] by Ron Paul
% - [Money, blockchains, and social scalability][social-scalability] by Nick Szabo
%
% [social-scalability]: https://unenumerated.blogspot.co.at/2017/02/money-blockchains-and-social-scalability.html
%
