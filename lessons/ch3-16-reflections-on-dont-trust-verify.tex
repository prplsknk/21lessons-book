\chapter{Remarques sur \enquote{Ne vous fiez pas, vérifiez}}
\label{les:16}

\begin{chapquote}{Lewis Carroll, \textit{Les aventures d’Alice sous terre}}
\enquote{Préparez-vous à entendre les témoignages,} dit le Roi, \enquote{et
ensuite la sentence !}
\end{chapquote}

Bitcoin cherche à remplacer, ou tout du moins fournir une alternative aux
monnaies conventionnelles. Ces monnaies sont liées à une autorité centrale, peu
importe que l'on parle d'un cours légal comme le dollar américain ou des V-Bucks
de Fortnite. Dans les deux cas, vous êtes tenus de faire confiance à une
autorité centrale pour émettre, gérer et faire circuler votre argent. Bitcoin
rompt ce lien et résout cette préoccupation principale : la question de la
\textit{confiance}.

\begin{quotation}\begin{samepage}
\enquote{Le problème fondamental des monnaies traditionnelles, c'est toute la
confiance nécessaire à son fonctionnement. [...] Ce dont nous avons besoin,
c'est d'un système de paiement électronique basé sur la preuve cryptographique
au lieu de la confiance}
\begin{flushright} -- Satoshi Nakamoto\footnote{Satoshi Nakamoto, annonce
officielle de Bitcoin~\cite{bitcoin-announcement} et livre
blanc~\cite{whitepaper}}
\end{flushright}\end{samepage}\end{quotation}

Bitcoin remédie au problème de la confiance en étant totalement décentralisé,
sans serveur central ni parties de confiance. Pas seulement sans
\textit{tierces} parties de confiance, mais sans parties de confiance tout
court. Quand il n'y a pas d'autorité centrale, il n'y a \textit{personne} à qui
faire confiance, tout simplement. L'innovation c'est la décentralisation totale.
C'est la source de la ténacité de Bitcoin, la raison pour laquelle il est
toujours en vie. C'est aussi ce qui explique pourquoi nous avons du minage, des
nœuds, des portefeuilles physiques et oui, la blockchain. La seule chose à
laquelle vous devez faire \enquote{confiance}, c'est le fait que notre
connaissance des mathématiques et de la physique n'est pas totalement à côté de
la plaque et que la majorité des mineurs sont honnêtes (et ils sont incités à
l'être).

Tandis que le monde normal part du postulat \textit{\enquote{fiez vous, mais
vérifiez}}, Bitcoin se fonde sur le postulat \textit{\enquote{ne vous fiez pas,
vérifiez}}. Satoshi a très clairement insisté sur l'importance de se passer de
la confiance à la fois dans l'introduction et la conclusion du livre blanc de
Bitcoin.

\begin{quotation}\begin{samepage}
\enquote{Conclusion : nous avons proposé un système de transactions
électroniques se passant de confiance.}
\begin{flushright} -- Satoshi Nakamoto\footnote{Satoshi Nakamoto, livre blanc de
Bitcoin~\cite{whitepaper}}
\end{flushright}\end{samepage}\end{quotation}

Notez que \textit{se passant de confiance} est utilisé dans un contexte très
particulier. Nous parlons des tierces parties de confiance, c'est-à-dire
d'autres entités à qui vous vous fiez pour produire, détenir et traiter votre
argent. On partira par exemple du principe que vous pouvez faire confiance à
votre ordinateur.

Comme Ken Thompson l'a montré dans sa conférence au Turing Award, la confiance
est une chose extrêmement délicate en informatique. Quand vous lancez un
programme, vous devez vous fier à toutes sortes de logiciels (et de matériels)
qui, en théorie, pourraient modifier ce programme par malveillance. Pour citer
Thompson dans \textit{Remarques sur la confiance envers la confiance} :
\enquote{La morale est évidente. Vous ne pouvez pas faire confiance à du code
que vous n'avez pas entièrement écrit.}~\cite{trusting-trust}

\begin{figure}
  \includegraphics{assets/images/ken-thompson-hack.png}
  \caption{Extraits de l'article de Ken Thompson `Remarques sur la confiance
  envers la confiance'}
  \label{fig:ken-thompson-hack}
\end{figure}

Thompson a démontré que même si vous avez accès au code source, votre
compilateur --- ou tout autre programme ou matériel d'exécution --- pourrait
être corrompu et que la détection de cette porte dérobée serait très délicate.
Par conséquent, en pratique, un système sans aucun \textit{besoin de confiance}
n'existe pas. Pour ça vous auriez à créer tous vos logiciels (assembleurs,
compilateurs, éditeurs de liens, etc.) \textit{et} tout votre matériel de bout
en bout, sans l'aide d'un quelconque logiciel externe ou machine programmable.

\begin{quotation}\begin{samepage}
\enquote{Si vous voulez faire une tarte aux pommes à partir de rien, vous devez
d'abord inventer l'univers.}
\begin{flushright} -- Carl Sagan\footnote{Carl Sagan, \textit{Cosmos}
\cite{cosmos}}
\end{flushright}\end{samepage}\end{quotation}

Le piratage de Ken Thompson consiste en une porte dérobée particulièrement
astucieuse et difficile à détecter, qui fonctionne sans modifier aucun logiciel.
Des chercheurs ont trouvé le moyen de corrompre du matériel critique à la
sécurité en manipulant la polarité des impuretés dans le
silicium.~\cite{becker2013stealthy} Ils ont alors été capables de compromettre
un générateur cryptographique de nombres aléatoires, juste en modifiant les
propriétés physiques du truc dont sont faites les puces électroniques. Et
puisque cette modification est invisible, la porte dérobée est indétectable à la
vérification optique, l'un des mécanismes de détection de sabotage les plus
utilisés pour ce genre de puces.

\begin{figure}
  \includegraphics{assets/images/stealthy-hardware-trojan.png}
  \caption{Chevaux de Troie matériels furtifs de niveau dopant par Becker,
  Regazzoni, Paar, Burleson}
  \label{fig:stealthy-hardware-trojan}
\end{figure}

Ça vous fait peur ? Eh bien, même si vous étiez capable de tout fabriquer et
programmer à partir de rien, vous devriez quand même faire confiance aux
mathématiques sous-jacentes. Il vous faudrait être convaincu que
\textit{secp256k1} est une courbe elliptique sans porte dérobée. Oui, des portes
dérobées malveillantes peuvent être insérées dans les fondations mathématiques
des fonctions cryptographiques et c'est vraisemblablement déjà arrivé au moins
une fois.~\cite{wiki:Dual_EC_DRBG} Il y a de quoi être paranoïaque pour de
bonnes raisons et le fait que tout, depuis votre matériel, en passant par vos
logiciels, jusqu'aux courbes elliptiques que l'on utilise, peut receler une
porte dérobée~\cite{wiki:backdoors} en fait partie.

\begin{quotation}\begin{samepage}
\enquote{Ne vous fiez pas. Vérifiez.}
\begin{flushright} -- Des bitcoiners, un peu partout
\end{flushright}\end{samepage}\end{quotation}

Les exemples ci-dessus ont pour but d'illustrer que l'informatique \textit{sans
confiance} est utopique. Bitcoin est sans doute le seul système qui effleure
cette utopie et pourtant, il est \textit{à confiance réduite} --- visant à
l'éliminer partout où c'est possible. On peut soutenir que la chaîne de
confiance est infinie, puisque vous devrez également vous convaincre que les
calculs demandent de l'énergie, que P n'est pas égal à NP et que vous vivez bien
dans la réalité et pas dans une simulation orchestrée par des acteurs
malveillants.

Des développeurs travaillent sur des outils et des procédures cherchant à
minimiser encore plus toute confiance restante. Par exemple, les développeurs de
Bitcoin ont créé Gitian\footnote{\url{https://gitian.org/}}, qui est une méthode
de diffusion de logiciels permettant de faire des compilations déterministes.
L'idée, c'est que les chances de manipulation malveillante sont réduites lorsque
plusieurs développeurs parviennent à reproduire des exécutables identiques. Mais
les portes dérobées imaginaires ne sont pas le seul vecteur d'attaque. Un simple
chantage ou de l'extorsion sont des menaces bien réelles. Comme dans le
protocole principal, la décentralisation sert à réduire la confiance nécessaire.

Divers efforts sont déployés afin de résoudre le problème de l'amorçage,
similaire à celui de l'œuf et de la poule, brillamment mis en évidence par le
piratage de Ken Thompson~\cite{web:bootstrapping}.
Guix\footnote{\url{https://guix.gnu.org}} (à prononcer \textit{geeks}), qui
utilise une gestion de paquets fonctionnellement déclarés et permet dès la
conception une compilation reproductible bit à bit, est un de ces efforts. Il en
résulte que vous n'avez plus à faire confiance aux serveurs qui vous fournissent
les logiciels, puisque vous pouvez vérifier par vous-même que l'exécutable
fourni est intact en le recompilant de zéro. Une \textit{pull request} a
récemment été fusionnée afin d'intégrer Guix dans le procédé de compilation de
Bitcoin.\footnote{Voir la PR 15277 de \texttt{bitcoin-core}: \\
\url{https://github.com/bitcoin/bitcoin/pull/15277}}

\begin{figure}
  \includegraphics{assets/images/guix-bootstrap-dependencies.png}
  \caption{Qui était là le premier, l'œuf ou la poule ?}
  \label{fig:guix-bootstrap-dependencies}
\end{figure}

Par chance, Bitcoin ne se repose pas sur un seul algorithme ou une seule sorte
de matériel. Une conséquence de la décentralisation radicale de Bitcoin, c'est
un modèle de sécurité distribué. Bien que les portes dérobées précédemment
décrites doivent être prises au sérieux, il est improbable que chaque
portefeuille logiciel, chaque portefeuille matériel, chaque bibliothèque de
fonctions cryptographiques, chaque implémentation de nœud et chaque compilateur
de chaque langage soient compromis. C'est possible, mais très hautement
improbable.

Notez par ailleurs que vous pouvez très bien générer une clé privée sans même
vous servir d'un quelconque logiciel ou matériel. Vous pouvez tirer un certain
nombre de fois à pile ou face~\cite{antonopoulos2014mastering}, mais selon votre
style de lancer cette source d'aléatoire ne le sera peut-être pas assez. Ce
n'est pas pour rien que des protocoles de stockage comme
Glacier\footnote{\url{https://glacierprotocol.org/}} recommandent d'utiliser un
dé de qualité casino comme une des deux sources d'entropie.

J'ai été forcé par Bitcoin à réfléchir à ce qu'implique réellement de ne se fier
à personne. Il m'a sensibilisé au problème de l'amorçage et de la chaîne de
confiance implicite dans le développement et l'exécution des programmes. Il m'a
fait également prendre conscience des multiples manières de compromettre un
logiciel ou un matériel.

\paragraph{Bitcoin m'a appris à ne pas me fier, mais à vérifier.}

% ---
%
% #### Down the Rabbit Hole
%
% - [The Bitcoin whitepaper][Nakamoto] by Satoshi Nakamoto
% - [Reflections on Trusting Trust][\textit{Reflections on Trusting Trust}] by Ken Thompson
% - [51% Attack][majority] on the Bitcoin Developer Guide
% - [Bootstrapping][bootstrapping], Guix Manual
% - [Secp256k1][secp256k1] on the Bitcoin Wiki
% - [ECC Backdoors][backdoors], [Dual EC DRBG][has already happened] on Wikipedia
%
% [Emmanuel Boutet]: https://commons.wikimedia.org/wiki/User:Emmanuel.boutet
% [\textit{Reflections on Trusting Trust}]: https://www.archive.ece.cmu.edu/~ganger/712.fall02/papers/p761-thompson.pdf
% [found a way]: https://scholar.google.com/scholar?hl=en&as_sdt=0%2C5&q=Stealthy+Dopant-Level+Hardware+Trojans&btnG=
% [Gitian]: https://gitian.org/
% [bootstrapping]: https://www.gnu.org/software/guix/manual/en/html_node/Bootstrapping.html
% [Guix]: https://www.gnu.org/software/guix/
% [pull-request]: https://github.com/bitcoin/bitcoin/pull/15277
% [flip a coin]: https://github.com/bitcoinbook/bitcoinbook/blob/develop/ch04.asciidoc#private-keys
% [Glacier]: https://glacierprotocol.org/
% [secp256k1]: https://en.bitcoin.it/wiki/Secp256k1
% [majority]: https://bitcoin.org/en/developer-guide#term-51-attack
%
% <!-- Wikipedia -->
% [backdoors]: https://en.wikipedia.org/wiki/Elliptic-curve_cryptography#Backdoors
% [has already happened]: https://en.wikipedia.org/wiki/Dual_EC_DRBG
% [Carl Sagan]: https://en.wikipedia.org/wiki/Cosmos_%28Carl_Sagan_book%29
% [alice]: https://en.wikipedia.org/wiki/Alice%27s_Adventures_in_Wonderland
% [carroll]: https://en.wikipedia.org/wiki/Lewis_Carroll
