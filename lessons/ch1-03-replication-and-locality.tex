\chapter{Réplication et localité}
\label{les:3}

\begin{chapquote}{Lewis Carroll, \textit{Alice au pays des merveilles}}
Ensuite résonna une voix furieuse, celle du Lapin, en train de crier :
\enquote{Pat ! Pat ! Où es-tu ?}
\end{chapquote}

Si l'on met de côté la mécanique quantique, la localité au sein du monde
physique n'est pas un problème. La question \textit{\enquote{Où se trouve X ?}}
trouve une réponse sensée, peu importe si X est une personne ou un objet. Dans
le monde numérique, la question du \textit{où} est déjà délicate en soi mais on
peut potentiellement y répondre. Sans rire, où sont situés vos e-mails ?
\enquote{Le cloud} serait une mauvaise réponse, c'est juste l'ordinateur de
quelqu'un d'autre. Pourtant, si vous vouliez situer chaque stockage contenant
une copie de vos e-mails, en théorie, vous pourriez.

Avec Bitcoin, la question du \enquote{où} est \textit{vraiment} délicate. Où
sont situés vos bitcoins, exactement ?

\begin{quotation}\begin{samepage}
\enquote{J'ai ouvert les yeux, regardé autour de moi et j'ai posé l'inévitable,
la sempiternelle, la tristement banale question postopératoire : `Où suis-je ?'}
\begin{flushright} -- Daniel Dennett\footnote{Daniel Dennett, \textit{Where Am
I?}~\cite{where-am-i}}
\end{flushright}\end{samepage}\end{quotation}

C'est un double problème : d'abord, le registre distribué l'est par réplication
totale, ce qui signifie que le registre est partout. Deuxièmement, les bitcoins
n'existent pas. Pas seulement physiquement, non, \textit{techniquement} aussi.

Bitcoin gère un ensemble de transactions sortantes non-dépensées, sans jamais
devoir mentionner une quelconque entité représentant un bitcoin.
L'existence d'un bitcoin se déduit en observant cet ensemble de transactions,
tout en désignant comme bitcoin chaque entrée totalisant 100 millions d'unités
de base.

\begin{quotation}\begin{samepage}
\enquote{Où est-il pendant le transfert, à ce moment ? [...] Premièrement, il
n'y a pas de bitcoins. Il n'y en a simplement pas. Ils n'existent pas. Il y a
des entrées dans un registre qui est partagé [...] Il n'existent dans aucun
lieu physique. Le registre lui existe partout, en gros. La géographie n'a pas de
sens ici ; ça ne vous aidera pas à définir votre politique.}
\begin{flushright} -- Peter Van Valkenburgh\footnote{Peter Van Valkenburgh dans
le podcast \textit{What Bitcoin Did}, épisode 49 \cite{wbd049}}
\end{flushright}\end{samepage}\end{quotation}

Par conséquent, que détenez-vous vraiment lorsque vous dites
\textit{\enquote{j'ai un bitcoin}}, s'ils n'existent pas ? Eh bien, vous vous
souvenez de tous ces mots étranges que le portefeuille que vous utilisez vous a
forcé à écrire ? Ce que vous détenez ce sont justement ces mots sorciers : une
formule magique\footnote{The Magic Dust of Cryptography: Comment l'information
numérique change notre société \cite{gigi:magic-spell}} qui sert à ajouter des
entrées dans le registre public ; les clés qui permettent de \enquote{déplacer}
des bitcoins. À toutes fins utiles, c'est pour ça que vos clés privées
\textit{sont} vos bitcoins. Si vous vous dites que j'invente tout ça, n'hésitez
pas à m'envoyer vos clés privées.

\paragraph{Bitcoin m'a appris que la localité était une histoire complexe.}

% ---
%
% #### Through the Looking-Glass
%
% - [The Magic Dust of Cryptography: How digital information is changing our society][a magic spell]
%
% #### Down the Rabbit Hole
%
% - [Where Am I?][Daniel Dennett] by Daniel Dennett
% - 🎧 [Peter Van Valkenburg on Preserving the Freedom to Innovate with Public Blockchains][wbd049] WBD #49 hosted by Peter McCormack
%
% <!-- Through the Looking-Glass -->
% [a magic spell]: 
%
% <!-- Down the Rabbit Hole -->
% [Daniel Dennett]: https://www.lehigh.edu/~mhb0/Dennett-WhereAmI.pdf
% [1st Amendment]: https://en.wikipedia.org/wiki/First_Amendment_to_the_United_States_Constitution
% [wbd049]: https://www.whatbitcoindid.com/podcast/coin-centers-peter-van-valkenburg-on-preserving-the-freedom-to-innovate-with-public-blockchains
%
% <!-- Wikipedia -->
% [alice]: https://en.wikipedia.org/wiki/Alice%27s_Adventures_in_Wonderland
% [carroll]: https://en.wikipedia.org/wiki/Lewis_Carroll
