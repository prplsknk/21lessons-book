\chapter{Une monnaie saine}
\label{les:14}

\begin{chapquote}{Lewis Carroll, \textit{Alice au pays des merveilles}}
\enquote{La première chose que je dois faire,} se dit-elle tout en marchant dans
le bois à l’aventure, \enquote{c’est retrouver ma taille normale ; la seconde,
c’est de trouver le chemin qui mène à ce charmant jardin. Je crois que c’est un
très bon plan.}
\end{chapquote}

La leçon la plus importante que j'ai tirée de Bitcoin c'est qu'en fin de compte,
la monnaie forte est meilleure que la monnaie faible. La monnaie forte,
également appelée \textit{monnaie saine}, consiste en toute monnaie échangeable
sur le marché mondial pouvant servir de réserve de valeur solide.

D'accord, Bitcoin est encore jeune et volatil. Les critiques avanceront qu'il
n'est pas fiable en tant que réserve de valeur. Mais l'argument de la volatilité
passe à côté du sujet. Il faut s'attendre à de la volatilité. Ça prendra un
moment au marché pour déterminer le juste prix de cette nouvelle monnaie. De
plus, il est fondé sur une erreur de métrique, comme le souligne une
plaisanterie récurrente. Si vous réfléchissez en dollars, vous passerez à côté
du fait qu'un bitcoin vaudra toujours un bitcoin.

\begin{quotation}\begin{samepage}
\enquote{Une offre monétaire fixe, ou une offre modifiée uniquement sur la base
de critères objectifs et calculables, est une condition nécessaire à un prix de
la monnaie juste et significatif.}
\begin{flushright} -- Père Bernard W. Dempsey, S.J.\footnote{Perry J. Roets,
S.J., \textit{Revue de l'économie sociale} \cite{review-social-economy}}
\end{flushright}\end{samepage}\end{quotation}

\newpage

Comme l'a montré une courte ballade dans le cimetière des monnaies disparues,
l'argent qui peut être créé le sera. Dans l'Histoire, aucun être humain n'a su
jusqu'à présent résister à la tentation.

Bitcoin élimine cette tentation de la création monétaire de façon astucieuse.
Satoshi avait conscience de notre cupidité et de notre faillibilité --- c'est
pour cela qu'il a choisi une chose plus fiable que le contrôle humain : les
mathématiques.

\begin{figure}
  \centering
  \begin{equation}
  \sum\limits_{i=0}^{32} \frac{210000 \lfloor \frac{50*10^8}{2^i} \rfloor}{10^8}
  \end{equation}
  \caption{Formule de l'offre de Bitcoin}
  \label{fig:supply-formula-white}
\end{figure}

Bien que la formule ci-dessus soit utile pour décrire l'offre de Bitcoin, on ne
la trouve nulle part dans le code. L'émission de nouveaux bitcoins est contrôlée
par un algorithme, qui réduit tous les quatre ans~\cite{btcwiki:supply} la
récompense payée aux mineurs. Cette formule permet donc de résumer rapidement ce
qui se passe sous le capot. On comprend mieux ce qui se passe vraiment en
regardant la variation des récompenses de bloc, payées à quiconque trouve un
bloc valide, ce qui arrive à peu près toutes les 10 minutes.

\begin{figure}
  \includegraphics{assets/images/you-are-here.png}
  \caption{L'offre contrôlée de Bitcoin}
  \label{fig:you-are-here.png}
\end{figure}

Les formules, les fonctions logarithmiques et les exponentielles ne sont pas
particulièrement intuitives à comprendre. Le concept de \textit{sain} peut
s'appréhender plus facilement si on le voit autrement. Une fois que l'on sait
combien il existe d'une chose et que l'on sait combien cette chose est
difficile à produire ou à obtenir, nous comprenons immédiatement sa valeur. Ce
qui est vrai avec un tableau de Picasso, une guitare d'Elvis Presley ou un
violon de Stradivarius est également vrai pour la monnaie fiduciaire, l'or et
les bitcoins. 

La dureté des monnaies fiduciaires dépend des responsables de leurs planches à
billets respectives. Certains gouvernements seront sans doute plus enclins à
créer de plus larges quantités de monnaie que d'autres, aboutissant à une
monnaie plus faible. D'autres gouvernements seront plus modérés sur leur
émission monétaire, entraînant une monnaie plus forte.

\begin{samepage}\begin{quotation}
\enquote{Un aspect important de cette nouvelle réalité est que les institutions
telles que la réserve fédérale ne peuvent pas faire faillite. Elles peuvent
créer autant d'argent qu'elles en ont besoin pour un coût virtuellement nul.}
\begin{flushright} -- Jörg Guido Hülsmann\footnote{Jörg Guido Hülsmann,
\textit{L'éthique de la création monétaire}~\cite{hulsmann2008ethics}}
\end{flushright}\end{quotation}\end{samepage}

Avant que nous n'ayons des monnaies fiduciaires, la dureté de l'argent était
déterminée par les propriétés naturelles de ce qui servait de monnaie. La
quantité d'or sur Terre est limitée par les lois de la physique. L'or est rare
car les collisions de supernovas et d'étoiles à neutrons sont rare. Le
\enquote{flux} d'or est limité car il demande des efforts à extraire. Comme
c'est un élément lourd, il est en majorité enterré bien profondément dans le
sol.

L'abolition de l'étalon-or a engendré une nouvelle réalité : il suffit d'un peu
d'encre pour créer de l'argent. Dans le monde actuel, ça demande encore moins
d'efforts d'ajouter quelques zéros au solde d'un compte bancaire : il suffit de
modifier quelques octets sur l'ordinateur d'une banque.

On peut exprimer plus généralement le principe énoncé ci-dessus comme étant le
rapport entre les \enquote{stocks} et les \enquote{flux}. Plus simplement, le
\textit{stock} représente la quantité existante de quelque chose. Pour nos
besoins, le stock est la mesure de l'offre actuelle de monnaie. Le \textit{flux}
quantifie la production de cette même chose sur une durée donnée (par an, par
exemple). La clé pour comprendre la monnaie saine est de comprendre le rapport
stock-à-flux.

Calculer le rapport stock-à-flux de la monnaie fiduciaire est complexe, car
l'offre de monnaie dépend de ce que vous y intégrez~\cite{wiki:money-supply}.
Vous pouvez ne compter que les billets et les pièces (M0), ajouter les chèques
de voyage et les remises de chèques (M1), ajouter les comptes épargne, les fonds
communs et quelques autres trucs (M2) et même ajouter à tout ça les certificats
de dépôt (M3). De plus, la façon dont tout cela est défini et calculé dépend de
chaque pays et puisque la réserve fédérale des États-Unis a cessé de publier
\cite{web:fed-m3} les chiffres pour M3, nous allons devoir faire avec l'offre
monétaire M2. J'adorerais pouvoir vérifier ces chiffres, mais pour le moment
j'imagine que nous devrons faire confiance à la réserve fédérale.

C'est l'or, l'un des métaux les plus rares sur Terre, qui a le rapport
stock-à-flux le plus élevé. Selon l'Institut d'études géologiques des
États-Unis, un peu plus de 190 000 tonnes en ont été minées au total. Au cours
des dernières années, environ 3100 tonnes ont été minées par
an~\cite{mineral-commodity-summaries}.

À partir de ces chiffres, nous pouvons facilement calculer le rapport
stock-à-flux de l'or (voir la Figure~\ref{fig:stock-to-flow-gold}).

\begin{figure}
  \centering
  \begin{equation}
  \frac{190,000 t}{3,100 t} = ~ 61
  \end{equation}
  \caption{Rapport stock-à-flux de l'or}
  \label{fig:stock-to-flow-gold}
\end{figure}

Il n'y a rien qui ait un rapport stock-à-flux plus élevé. C'est pour cette
raison que l'or, jusqu'à maintenant, était la monnaie la plus forte et la plus
saine qui soit. On raconte souvent que tout l'or qui a déjà été miné tiendrait
dans deux piscines olympiques. Selon mes
calculs\footnote{\url{https://bit.ly/gold-pools}}, il en faudrait quatre. Donc
soit les piscines olympiques ont rétréci, soit il faudrait peut-être revoir ça.

Arrive alors le Bitcoin. Comme vous le savez sans doute déjà, le minage de
bitcoin fait fureur depuis quelques années. Cela s'explique car nous sommes
encore au début de ce qu'on appelle \textit{le temps des récompenses}, où les
nœuds de minage sont récompensés avec \textit{beaucoup} de bitcoin pour leurs
efforts de calcul. Nous sommes en ce moment dans l'époque numéro 3, qui a débuté
en 2016 et s'achèvera au début de 2020, sans doute en mai. Tandis que l'offre de
bitcoins est limitée, les rouages internes de Bitcoin ne permettent d'établir
que des dates approximatives. Pourtant, nous pouvons prédire avec certitude à
quel niveau le rapport stock-à-flux de Bitcoin se situera. Alerte spoiler : ça
sera élevé.

Élevé comment ? Eh bien, il s'avère que Bitcoin finira par devenir infiniment
fort (voir la Figure~\ref{fig:stock-to-flow-white-cropped}).

\begin{figure}
  \includegraphics{assets/images/stock-to-flow-white-cropped.png}
  \caption{Visualisation du stock et du flux du dollar US, de l'or et de
  Bitcoin}
  \label{fig:stock-to-flow-white-cropped}
\end{figure}

\paragraph{}
À cause de la réduction exponentielle des récompenses de minage, le flux de
nouveaux bitcoins va diminuer, engendrant un rapport stock-à-flux qui grimpe en
flèche. Il rattrapera l'or en 2020, pour mieux le surpasser quatre ans plus tard
en doublant à nouveau sa dureté. Au total, un tel doublement se produira 64
fois. Grâce à la puissance des exponentielles, le nombre de bitcoins minés par
an tombera à moins de 100 bitcoins dans 50 ans et à moins de 1 bitcoin dans 75
ans. Le robinet mondial que représentent les récompenses de bloc se tarira aux
environs de l'année 2140, mettant effectivement un terme à la production de
bitcoin. C'est un jeu de longue haleine. Si vous lisez ceci, vous êtes encore en
avance.

\begin{figure}
  \includegraphics{assets/images/soundness-over-time.png}
  \caption{Le rapport stock-à-flux du bitcoin comparé à l'or}
  \label{fig:soundness-over-time}
\end{figure}

Alors que le bitcoin tend vers un rapport stock-à-flux infini, il deviendra la
monnaie la plus saine qui soit. La dureté infinie semble difficile à battre.

D'un point de vue économique, \textit{l'ajustement de la difficulté} de Bitcoin
est son aspect le plus important. La difficulté à miner du bitcoin dépend de la
rapidité avec laquelle de nouveaux bitcoins sont minés\footnote{En réalité ça
dépend de la vitesse à laquelle des blocs valides sont trouvés, mais pour nos
besoins, c'est la même chose que de \enquote{miner des bitcoins} et ça le
restera pour les 120 prochaines années.}. C'est l'ajustement dynamique de la
difficulté de minage du réseau qui nous permet de prédire son offre future.

La simplicité de l'algorithme d'ajustement de la difficulté pourrait détourner
de sa profondeur, mais il est véritablement une révolution aux proportions
dignes d'Einstein. Il garantit que quels que soient les efforts déployés dans le
minage, l'offre maîtrisée de Bitcoin ne sera pas perturbée. À la différence de
toutes les autres ressources, peu importe l'énergie dépensée par quelqu'un dans
le minage de bitcoin, la récompense totale n'augmentera pas.

Tout comme $E=mc^2$ impose une limite universelle à la vitesse dans notre
univers, l'ajustement de la difficulté de minage impose sa \textbf{limite
monétaire universelle} à Bitcoin.

\paragraph{}
Sans cet ajustement de la difficulté, tous les bitcoins auraient déjà été minés.
Sans cet ajustement de la difficulté, Bitcoin n'aurait probablement pas survécu
à ses premiers pas. C'est ce qui sécurise le réseau durant le temps des
récompenses. C'est ce qui garantit une distribution stable et
impartiale\footnote{Dan Held, \textit{La distribution de Bitcoin était
juste}~\cite{distribution-was-fair}} des nouveaux bitcoins. C'est le thermostat
qui régule la politique monétaire de Bitcoin.

Einstein nous a enseigné une chose novatrice : peu importe la force imprimée à
un objet, à un moment donné vous ne pourrez pas le faire aller plus vite.
Satoshi nous a aussi enseigné une chose novatrice : peu importent les efforts
mis dans le minage de cet or numérique, à un moment donné vous ne pourrez pas en
tirer plus de bitcoins. Pour la première fois dans l'Histoire de l'Humanité,
nous avons un bien monétaire dont vous ne pourrez pas augmenter la production,
peu importe à quel point vous essaierez.

\paragraph{Bitcoin m'a appris que la monnaie saine était indispensable.}

% ---
%
% #### Through the Looking-Glass
%
% - [Bitcoin's Energy Consumption: A Shift in Perspective][much energy]
%
% #### Down the Rabbit Hole
%
% - [The Ethics of Money Production][Jörg Guido Hülsmann] by Jörg Guido Hülsmann
% - [Mineral Commodity Summaries 2019][last few years] by the United States Geological Survey
% - [Bitcoin’s Distribution was Fair][fair distribution] by Dan Held
% - [Bitcoin's Controlled Supply][algorithmically controlled] on the Bitcoin Wiki
% - [Money Supply][how much money there is], [Speed of Light][universal speed limit] on Wikipedia
%
% <!-- Internal -->
% [much energy]: 
%
% [Fr. Bernard W. Dempsey, S.J.]: https://www.jstor.org/stable/29769582
% [Jörg Guido Hülsmann]: https://mises.org/sites/default/files/The%20Ethics%20of%20Money%20Production_2.pdf
% [stopped publishing]: https://www.federalreserve.gov/Releases/h6/discm3.htm
% [last few years]: https://minerals.usgs.gov/minerals/pubs/mcs/2018/mcs2018.pdf
% [my calculations]: https://www.wolframalpha.com/input/?i=volume+of+190000+metric+tons+gold+%2F+olympic+swimming+pool+volume
% [fair distribution]: https://blog.picks.co/bitcoins-distribution-was-fair-e2ef7bbbc892
%
% <!-- Bitcoin Wiki -->
% [algorithmically controlled]: https://en.bitcoin.it/wiki/Controlled_supply
%
% <!-- Wikipedia -->
% [how much money there is]: https://en.wikipedia.org/wiki/Money_supply
% [universal speed limit]: https://en.wikipedia.org/wiki/Speed_of_light#Upper_limit_on_speeds
% [alice]: https://en.wikipedia.org/wiki/Alice%27s_Adventures_in_Wonderland
% [carroll]: https://en.wikipedia.org/wiki/Lewis_Carroll
