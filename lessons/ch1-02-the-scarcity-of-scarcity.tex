
\chapter{La rareté de la rareté}
\label{les:2}

\begin{chapquote}{Alice}
\enquote{Cela suffit comme cela\ldots J’espère que je ne grandirai plus\ldots}
\end{chapquote}

Généralement, le progrès technologique semble rendre les choses plus abondantes.
Ce qui était auparavant un produit de luxe devient accessible à de plus en plus
de gens. Bientôt, nous vivrons tous comme des rois. C'est déjà le cas pour la
plupart d'entre nous. Comme l'écrivait Peter Diamandis dans
Abundance~\cite{abundance} : \enquote{La technologie est un mécanisme de
libération des ressources. Elle peut rendre abondant ce qui était rare.}

Bitcoin, en tant que technologie avancée, casse cette tendance et crée une
nouvelle ressource authentiquement rare. Certains avancent même que c'est l'une
des ressources les plus rares de l'univers. L'offre ne peut pas grossir, quels
que soient les efforts déployés pour y parvenir.

\begin{quotation}\begin{samepage}
\enquote{Il n'y a que deux choses véritablement rares : le temps et Bitcoin.}
\begin{flushright} -- Saifedean Ammous\footnote{Présentation sur The Bitcoin
Standard~\cite{bitcoinstandard-pres}}
\end{flushright}\end{samepage}\end{quotation}

Paradoxalement, ceci se produit par un mécanisme de réplication. Les
transactions sont diffusées, les blocs se propagent, le registre distribué est
-- vous l'avez deviné -- distribué. Mais ce ne sont que des mots savants pour
désigner la copie. Bon sang, Bitcoin se réplique même tout seul sur autant
d'ordinateurs que possible, en incitant les gens à exécuter des nœuds complets
et à miner de nouveaux blocs.

Toute cette réplication œuvre magnifiquement de concert en vue de produire de la
rareté.

\paragraph{}

\paragraph{En ces temps d'abondance, Bitcoin m'a appris ce qu'était la véritable
rareté.}

% ---
%
% #### Through the Looking-Glass
%
% - [Lesson 14: Sound money][lesson14]
%
% #### Down the Rabbit Hole
%
% - [The Bitcoin Standard: The Decentralized Alternative to Central Banking][bitcoin-standard]
% - [Abundance: The Future Is Better Than You Think][Abundance] by Peter Diamandis
% - [Presentation on The Bitcoin Standard][bitcoin-standard-presentation] by Saifedean Ammous
% - [Modeling Bitcoin's Value with Scarcity][planb-scarcity] by PlanB
% - 🎧 [Misir Mahmudov on the Scarcity of Time & Bitcoin][tftc60] TFTC #60 hosted by Marty Bent
% - 🎧 [PlanB – Modelling Bitcoin's digital scarcity through stock-to-flow techniques][slp67] SLP #67 hosted by Stephan Livera
%
% <!-- Through the Looking-Glass -->
% [lesson14]: {{ 'bitcoin/lessons/ch2-14-sound-money' | absolute_url }}
%
% <!-- Down the Rabbit Hole -->
% [Abundance]: https://www.diamandis.com/abundance
% [bitcoin-standard]: http://amzn.to/2L95bJW
% [bitcoin-standard-presentation]: https://www.bayernlb.de/internet/media/de/ir/downloads_1/bayernlb_research/sonderpublikationen_1/bitcoin_munich_may_28.pdf
% [planb-scarcity]: https://medium.com/@100trillionUSD/modeling-bitcoins-value-with-scarcity-91fa0fc03e25
% [tftc60]: https://anchor.fm/tales-from-the-crypt/episodes/Tales-from-the-Crypt-60-Misir-Mahmudov-e3aibh
% [slp67]: https://stephanlivera.com/episode/67
%
% <!-- Wikipedia -->
% [alice]: https://en.wikipedia.org/wiki/Alice%27s_Adventures_in_Wonderland
% [carroll]: https://en.wikipedia.org/wiki/Lewis_Carroll
