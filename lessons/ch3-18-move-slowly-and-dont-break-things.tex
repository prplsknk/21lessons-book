\chapter{Avancer lentement sans rien casser}
\label{les:18}

\begin{chapquote}{Lewis Carroll, \textit{Les aventures d'Alice sous terre}}
Aussi la barque serpentait-elle doucement, sous le brillant jour d’été, avec son
joyeux équipage et sa musique de voix et d’éclats de rire\ldots
\end{chapquote}

J'enfonce peut-être des portes ouvertes, mais le monde de la tech fonctionne
toujours aujourd'hui en \enquote{avançant vite et en cassant des trucs}. L'idée
de ne pas s'efforcer à réussir du premier coup est une des bases de la mentalité
\textit{échoue tôt, échoue souvent}. Le succès se mesure à la croissance, donc
tant que vous grandissez, tout ira bien. Si quelque chose ne fonctionne pas
comme prévu vous n'avez qu'à pivoter et itérer. En d'autres termes : si vous
jetez assez de merde contre le mur, vous verrez bien ce qui colle.

Bitcoin est très différent. Il l'est dès sa conception. Il est différent par
besoin. Satoshi l'a dit lui-même, les monnaies électroniques ont été tentées à
de nombreuses reprises et tous ces essais ont échoué car il y avait une tête à
faire tomber. L'innovation de Bitcoin, c'est d'être un animal sans tête.

\begin{quotation}\begin{samepage}
\enquote{Beaucoup de gens refusent par réflexe l'idée des monnaies électroniques
à cause de toutes les entreprises qui ont échoué dans les années 90. J'espère
qu'il est clair que c'est la nature du contrôle centralisé de ces systèmes qui a
causé leur perte.}
\begin{flushright} -- Satoshi Nakamoto\footnote{Satoshi Nakamoto, dans une
réponse à Sepp Hasslberger \cite{satoshi-centralized-nature}}
\end{flushright}\end{samepage}\end{quotation}

Une des conséquences de cette décentralisation radicale est la résistance
inhérente au changement. \enquote{Avancer vite en cassant des trucs} ne
fonctionne et ne fonctionnera jamais sur la couche de base de Bitcoin. Même si
on le voulait, ce ne serait pas possible sans convaincre \textit{tout le monde}
de changer sa façon de faire. Voilà ce qu'est le consensus distribué. Voilà la
nature de Bitcoin.

\begin{quotation}\begin{samepage}
\enquote{La nature de Bitcoin est telle qu'une fois sortie la version 0.1, les
concepts essentiels étaient gravés dans le marbre pour le restant de ses jours.}
\begin{flushright} -- Satoshi Nakamoto\footnote{Satoshi Nakamoto, dans une
réponse à Gavin Andresen \cite{satoshi-centralized-nature}}
\end{flushright}\end{samepage}\end{quotation}

C'est l'une des nombreuses propriétés paradoxales de Bitcoin. Nous avons tous
fini par croire que tout logiciel peut facilement être modifié. Mais la nature
de cet animal rend tout changement sacrément difficile.

Comme Hasu l'écrit élégamment dans Décortiquer le contrat social de Bitcoin
\cite{social-contract}, on ne peut changer ses règles qu'en
\textit{proposant} un changement et par conséquent en \textit{convainquant} tous
les utilisateurs de Bitcoin de l'adopter. Bien qu'il soit un logiciel, cela rend
Bitcoin très résilient au changement.

Cette résilience est l'un des attributs de Bitcoin les plus importants. Les
systèmes logiciels critiques se doivent d'être anti-fragile, ce qui est garanti
par l'interaction entre les couches sociales et techniques de Bitcoin. Les
systèmes monétaires sont agressifs par nature et nous le savons depuis des
milliers d'années, un environnement agressif requiert des fondations solides.

\begin{quotation}\begin{samepage}
\enquote{La pluie est tombée, les torrents sont venus, les vents ont soufflé et
se sont jetés contre cette maison : elle n'est point tombée, parce qu'elle était
fondée sur le roc.}
\begin{flushright} -- Matthieu 7:25
\end{flushright}\end{samepage}\end{quotation}

Vraisemblablement, dans cette parabole des bâtisseurs sages et sots, Bitcoin
n'est pas la maison. C'est le roc. Invariable, stable, apportant le socle d'un
nouveau système financier.

À l'image des géologues qui savent que les formations rocheuses évoluent et sont
toujours en mouvement, on peut s'apercevoir que Bitcoin bouge et évolue. Il faut
juste savoir où et comment regarder.

L'introduction du \textit{pay to script hash}\footnote{Les transactions Pay to
Script Hash (P2SH) ont été normalisées dans le BIP16. Elles permettent aux
transactions d'être envoyées à un hash de script (adresse commençant par 3) au
lieu d'un hash d'adresse publique (adresse commençant par
1).~\cite{btcwiki:p2sh}} et de \textit{segregated witness}\footnote{Segregated
Witness (abrégé en SegWit) est une mise à jour implémentée du protocole qui vise
à fournir une protection envers la plasticité des transactions et à élargir la
taille des blocs. SegWit sépare le \textit{témoin} (witness en anglais) de la
liste des entrées.~\cite{btcwiki:segwit}} sont la preuve que les règles de
Bitcoin peuvent être changées tant qu'il y assez d'utilisateurs convaincus que
ce changement bénéficie au réseau. SegWit a permis le développement du réseau
Lightning\footnote{\url{https://lightning.network/}} qui est l'une des maisons
en construction sur le socle solide de Bitcoin. Des mises à jour futures comme
les signatures de Schnorr~\cite{bip:schnorr} amélioreront l'efficacité et la
confidentialité, ainsi que les scripts (comprendre: contrats intelligents) qui
seront indiscernables des transactions classiques grâce à
Taproot~\cite{taproot}. Les sages bâtisseurs construisent bel et bien sur des
fondations solides.

Satoshi n'était pas seulement un sage bâtisseur de technologie. Il comprenait
aussi la nécessité de prendre de sages décisions idéologiques.

\begin{quotation}\begin{samepage}
\enquote{Être open source veut dire que quiconque peut personnellement vérifier
le code. Si les sources étaient privées, personne ne pourrait vérifier la
sécurité. Je pense qu'un programme de cette nature doit obligatoirement être
open source.}
\begin{flushright} -- Satoshi Nakamoto\footnote{Satoshi Nakamoto, dans une
réponse à SmokeTooMuch \cite{satoshi-open-source}}
\end{flushright}\end{samepage}\end{quotation}

L'ouverture est primordiale à la sécurité et inhérente à l'open source et au
mouvement du logiciel libre. Comme Satoshi le faisait remarquer, les protocoles
sécurisés et le code qui les implémente doivent être ouverts --- l'obscurité ne
procure aucune sécurité. Encore une fois, un autre avantage est lié à la
décentralisation : du code qui peut être exécuté, lu, modifié, copié et
distribué librement garantit sa diffusion rapide et étendue.

La nature radicalement décentralisée de Bitcoin est ce qui lui permet de se
déplacer lentement et sciemment. Un réseau de nœuds, détenus par des
individus souverains, est intrinsèquement résistant au changement ---
malveillant ou pas. Sans possibilité de forcer des mises à jour, la seule façon
d'introduire des modifications est de convaincre lentement chacun des
participants de l'adopter. Ce procédé décentralisé de proposer et de déployer
les changements est ce qui rend le réseau extraordinairement résilient face aux
modifications malveillantes. C'est aussi ce qui rend les réparations plus
complexes que dans un environnement centralisé et qui explique que tout le monde
cherche avant tout à ne rien casser.

\paragraph{Bitcoin m'a appris qu'avancer lentement était une fonctionnalité, pas
un bug.}

% ---
%
% #### Through the Looking-Glass
%
% - [Lesson 1: Immutability and Change][lesson1]
%
% #### Down the Rabbit Hole
%
% - [Unpacking Bitcoin's Social Contract] by Hasu
% - [Schnorr signatures BIP][Schnorr signatures] by Pieter Wuille
% - [Taproot proposal][Taproot] by Gregory Maxwell
% - [P2SH][pay to script hash], [SegWit][segregated witness] on the Bitcoin Wiki
% - [Parable of the Wise and the Foolish Builders][Matthew 7:24--27] on Wikipedia
%
% <!-- Down the Rabbit Hole -->
% [lesson1]: {{ '/bitcoin/lessons/ch1-01-immutability-and-change' | absolute_url }}
%
% [Unpacking Bitcoin's Social Contract]: https://uncommoncore.co/unpacking-bitcoins-social-contract/
% [Matthew 7:24--27]: https://en.wikipedia.org/wiki/Parable_of_the_Wise_and_the_Foolish_Builders
% [pay to script hash]: https://en.bitcoin.it/wiki/Pay_to_script_hash
% [segregated witness]: https://en.bitcoin.it/wiki/Segregated_Witness
% [lightning network]: https://lightning.network/
% [Schnorr signatures]: https://github.com/sipa/bips/blob/bip-schnorr/bip-schnorr.mediawiki#cite_ref-6-0
% [Taproot]: https://lists.linuxfoundation.org/pipermail/bitcoin-dev/2018-January/015614.html
%
% <!-- Wikipedia -->
% [alice]: https://en.wikipedia.org/wiki/Alice%27s_Adventures_in_Wonderland
% [carroll]: https://en.wikipedia.org/wiki/Lewis_Carroll
