\chapter{L'inflation}
\label{les:9}

\begin{chapquote}{La Reine Rouge} 
\enquote{Ici, vois-tu, on est obligé de courir tant qu'on peut pour rester au
même endroit. Si on veut aller ailleurs, il faut courir au moins deux fois plus
vite que ça !}
\end{chapquote}

Tenter de comprendre l'inflation monétaire et comment un système
désinflationniste tel que Bitcoin pourrait changer nos comportements a constitué
le début de ma plongée dans l'économie. Je savais que l'inflation était le taux
auquel la monnaie était nouvellement émise, mais je n'en savais pas beaucoup
plus que ça.

Tandis que certains économistes prétendent que l'inflation est bonne, d'autres
prétendent qu'une monnaie \enquote{dure} qui ne peut être facilement produite
--- comme nous avions durant la période de l'étalon-or --- est indispensable à
une économie saine. Bitcoin, avec son offre fixe de 21 millions, partage l'avis
du second camp.

Habituellement, les effets de l'inflation ne sautent pas aux yeux. Selon le taux
d'inflation (ainsi que d'autres facteurs), le délai entre la cause et la
conséquence peut s'étendre sur plusieurs années. De plus, l'inflation touche
certains catégories plus que d'autres. Comme Henry Hazlitt le fait remarquer
dans \textit{L'Économie Politique en Une Leçon} : \enquote{L'art de la politique
économique consiste à ne pas considérer uniquement l'aspect immédiat d'un
problème ou d'un acte, mais à envisager ses effets plus lointains ; il consiste
essentiellement à considérer les conséquences que cette politique peut avoir,
non seulement sur un groupe d'hommes ou d'intérêts donnés, mais sur tous les
groupes existants.}

L'une de mes révélations fut le moment où j'ai compris que l'émission monétaire
--- imprimer plus d'argent --- était une activité économique \textit{totalement}
différente de toutes les autres activités économiques. Pendant que les vrais
biens et services produisent de la vraie valeur pour les vrais gens, imprimer
plus d'argent a l'exact effet contraire : on retire de la valeur à tous ceux
qui détiennent cette monnaie dont la quantité augmente.

\begin{quotation}\begin{samepage}
\enquote{L'inflation en soi --- c'est-à-dire la simple émission de plus de
monnaie, avec pour conséquences la hausse des salaires et l'accroissement des
prix --- peut très bien avoir l'air de créer une demande supplémentaire. Mais si
on raisonne en termes de production et d'échange des biens réels, il n'en est
rien.}
\begin{flushright} -- Henry Hazlitt\footnote{Henry Hazlitt, \textit{L'Économie
Politique en Une Leçon} \cite{hazlitt}}
\end{flushright}\end{samepage}\end{quotation}

La force destructrice de l'inflation devient évidente dès qu'un peu d'inflation
se change en \textit{beaucoup}. Si la monnaie subit une hyperinflation, les
choses tournent au vinaigre très rapidement
\footnote{\url{https://en.wikipedia.org/wiki/Hyperinflation}\cite{wiki:hyperinflation}}.
Au fur et à mesure qu'une monnaie se désagrège, elle échouera à stocker la
valeur à travers le temps et les gens se précipiteront pour mettre la main sur
n'importe quel bien qui pourra y parvenir.

\paragraph{}
Une autre conséquence de l'hyperinflation, c'est que tout l'argent épargné par
les gens durant leur vie va littéralement s'évaporer. L'argent liquide qui se
trouve dans votre portefeuille sera toujours là, bien entendu. Mais ça ne sera
effectivement plus que ça : du papier sans valeur.

\begin{figure}
  \includegraphics{assets/images/children-playing-with-money.png}
  \caption{Hyperinflation pendant la République de Weimar (1921-1923)}
  \label{fig:children-playing-with-money}
\end{figure}

\paragraph{}
La monnaie perd également de la valeur avec une inflation soi-disant
\enquote{modérée}. C'est juste qu'elle arrive suffisamment lentement pour que la
plupart des gens ne s'aperçoivent pas de la baisse de leur pouvoir d'achat. Et
dès lors que la planche à billets tourne, la quantité de monnaie peut être
facilement accrue et ce qui était auparavant une inflation modérée peut se
transformer par l'appui d'une simple bouton en une bonne dose d'inflation bien
forte. Friedrich Hayek le faisait remarquer dans l'un de ses essais, l'inflation
modérée mène généralement à l'inflation pure et simple.

\begin{quotation}\begin{samepage}
\enquote{Une inflation `modérée' et stable ne peut pas nous aider --- cela peut
seulement mener à l'inflation totale.}
\begin{flushright} -- Friedrich Hayek\footnote{Friedrich Hayek, \textit{Le
chômage des années 80 et les syndicats} \cite{hayek-inflation}}
\end{flushright}\end{samepage}\end{quotation}

L'inflation est particulièrement sournoise, puisqu'elle favorise ceux qui sont
au plus près du procédé d'émission. Il faut du temps pour que la monnaie
nouvellement émise circule et que les prix s'ajustent. Donc si vous avez la
possibilité de mettre la main sur plus d'argent avant que celui des autres ne se
dévalue, vous avez de l'avance sur la courbe d'inflation. C'est aussi pour ça
que l'on peut voir l'inflation comme un impôt caché, car au final ce sont les
gouvernements qui en profitent tandis que tout le monde en paye le prix.

\begin{quotation}\begin{samepage}
\enquote{Je ne pense pas que cela soit une exagération de dire que l’Histoire
est largement une histoire d’inflation, une inflation habituellement fabriquée
par les gouvernements, pour le gain des gouvernements.}
\begin{flushright} -- Friedrich Hayek\footnote{Friedrich Hayek, \textit{La Bonne
Monnaie} \cite{hayek-good-money}}
\end{flushright}\end{samepage}\end{quotation}

\paragraph{}
Jusqu'à présent, les devises contrôlées par les gouvernements ont toutes fini
par être remplacées ou s'effondrer totalement. Peu importe la faiblesse du taux
d'inflation, parler de croissance \enquote{stable} revient à parler de
croissance exponentielle. Dans la nature comme en économie, tous les systèmes
qui se développent de façon exponentielle devront se stabiliser sous peine de
subir un effondrement catastrophique.

\paragraph{}
\enquote{Ça ne peut pas arriver dans mon pays}, c'est probablement ce que vous
vous dites. Ce n'est pas du tout ce que vous vous dites si vous vivez au
Vénézuela, qui est en ce moment touché par l'hyperinflation. Avec un taux
d'inflation supérieur à un million de pourcents, leur argent ne vaut plus rien.
\cite{wiki:venezuela}

\paragraph{}
Ça pourrait encore prendre plusieurs années, ou ne pas toucher votre devise.
Mais il suffit de jeter un coup d'œil à la liste des anciennes
monnaies\footnote{Voir \textit{Liste des anciennes monnaies} sur Wikipedia.
\cite{wiki:historical-currencies}} pour voir que cela se produit invariablement,
au cours d'un temps suffisamment long. Je me souviens avoir réellement utilisé
toutes celles-ci : le schilling autrichien, le Deutsche mark, la lire italienne,
le franc français, la livre irlandaise, le dinar croate, etc. Ma grand-mère a
même utilisé la couronne austro-hongroise. Au fil du temps, les monnaies en
circulation\footnote{Voir \textit{Liste des monnaies en circulation} sur
Wikipedia \cite{wiki:list-of-currencies}} vont lentement mais sûrement se
diriger vers leurs cimetières respectifs. Elles subiront une hyperinflation ou
seront remplacées. Elles seront bientôt des monnaies anciennes. Nous les
rendrons obsolètes.

\begin{quotation}\begin{samepage}
\enquote{L'Histoire a montré que les gouvernements cèdent immanquablement à la
tentation de gonfler l'offre de monnaie.}
\begin{flushright} -- Saifedean Ammous\footnote{Saifedean Ammous,
\textit{L'Étalon Bitcoin} \cite{bitcoin-standard}}
\end{flushright}\end{samepage}\end{quotation}

\paragraph{}
Pourquoi Bitcoin est-il différent ? Contrairement aux monnaies imposées par les
états, les biens monétaires qui ne sont pas réglementés par des gouvernements,
mais par les lois de la physique\footnote{Gigi, \textit{La consommation
énergétique de Bitcoin - Une nouvelle perspective} \cite{gigi:energy}}, ont une
tendance à la survie et même à maintenir leur valeur au fil du temps. Jusqu'à
présent, le meilleur exemple est l'or qui, comme l'atteste le bien nommé
\textit{rapport or sur costume correct}\footnote{L'Histoire montre que le prix
d'une once d'or est égal au prix d'un costume pour homme de bonne facture, selon
le cabinet de conseil en investissement
Sionna\cite{web:gold-to-decent-suite-ratio}}, conserve sa valeur sur des
centaines et même des milliers d'années. Il n'est peut-être pas parfaitement
\enquote{stable} --- un concept discutable dès le départ --- mais la valeur
qu'il renferme reste au moins dans les mêmes ordres de grandeur.

Lorsqu'un bien monétaire ou une devise conserve efficacement sa valeur à travers
le temps et l'espace, il est perçu comme \textit{fort}. Si à l'inverse il ne
peut la maintenir, parce qu'il s'abîme ou gonfle facilement son offre, il est
considéré comme \textit{faible}. Le concept de dureté est essentiel à la
compréhension de Bitcoin et mérite un examen détaillé. Nous y reviendrons dans
la dernière leçon sur l'économie : la monnaie saine.

\paragraph{}
Alors que de plus en plus de pays souffrent d'hyperinflation, de plus en plus de
gens devront affronter la réalité des monnaies fortes et faibles. Si nous avons
de la chance, il se peut même que certaines banques centrales se trouvent
forcées de réévaluer leurs politiques monétaires. Quoi qu'il arrive, la lucidité
que m'a procuré Bitcoin s'avérera sans doute inestimable, quelle que soit
l'issue.

\paragraph{Bitcoin m'a appris que l'inflation était un impôt caché et que
l'hyperinflation était une catastrophe.}

% ---
%
% #### Down the Rabbit Hole
%
% - [Economics in One Lesson][Henry Hazlitt] by Henry Hazlitt
% - [1980's Unemployment and the Unions][unions] by Friedrich Hayek
% - [Good Money, Part II][good-money]: Volume Six of the Collected Works of F.A. Hayek
% - [The Bitcoin Standard] by Saifedean Ammous
% - [Hyperinflation][hyperinflates], [economic crisis in Venezuela][wiki-venezuela], [list of historical currencies], [list of currencies][currently in use] on Wikipedia
%
% [unions]: https://books.google.com/books/about/1980s_unemployment_and_the_unions.html?id=xM9CAQAAIAAJ
% [good-money]: https://books.google.com/books?id=l_A1vVIaYBYC
%
% [Henry Hazlitt]: https://mises.org/library/economics-one-lesson
% [hyperinflates]: https://en.wikipedia.org/wiki/Hyperinflation
% [inflation cannot help]: https://books.google.com/books?id=zZu3AAAAIAAJ&dq=%22only+while+it+accelerates%22&focus=searchwithinvolume&q=%22steady+inflation+cannot+help%22
% [history of inflation]: https://books.google.com/books?id=l_A1vVIaYBYC&pg=PA142&dq=%22history+is+largely+a+history+of+inflation%22&hl=en&sa=X&ved=0ahUKEwi90NDLrdnfAhUprVkKHUx1CmIQ6AEIKjAA#v=onepage&q=%22history%20is%20largely%20a%20history%20of%20inflation%22&f=false
% [wiki-venezuela]: https://en.wikipedia.org/wiki/Crisis_in_Venezuela#Economic_crisis
% [by the laws of physics]: https://link.medium.com/9fzq2L0J3S
% [\textit{Gold-to-Decent-Suit Ratio}]: https://www.businesswire.com/news/home/20110819005774/en/History-Shows-Price-Ounce-Gold-Equals-Price
% [The Bitcoin Standard]: https://thesaifhouse.wordpress.com/book/
%
% <!-- Wikipedia -->
% [alice]: https://en.wikipedia.org/wiki/Alice%27s_Adventures_in_Wonderland
% [carroll]: https://en.wikipedia.org/wiki/Lewis_Carroll
