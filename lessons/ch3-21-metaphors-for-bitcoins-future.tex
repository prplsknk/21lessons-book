\chapter{Métaphores pour le futur de Bitcoin}
\label{les:21}

\begin{chapquote}{Lewis Carroll, \textit{Alice au pays des merveilles}}
\enquote{Je sais qu’il arrive toujours quelque chose d’intéressant\ldots}
\end{chapquote}

Au cours des deux dernières décennies, il est devenu évident que l'innovation
technologique ne suivait pas une courbe linéaire. Que vous croyiez ou pas à la
singularité technologique, le progrès est indéniablement exponentiel dans de
nombreux domaines. En plus de ça, le taux d'adoption des technologies s'accélère
et sans vous en rendre compte, le buisson de la cour d'école du coin a disparu
car vos enfants utilisent Snapchat à la place. Les courbes exponentielles ont
cette tendance à vous exploser au visage bien avant que vous ne l'ayez vu venir.

Bitcoin est une technologie exponentielle reposant sur d'autres technologies
exponentielles. \textit{Our World in
Data}\footnote{\url{https://ourworldindata.org/}} montre admirablement la
vitesse croissante de l'adoption technologique à partir de 1903 avec l'arrivée
des lignes téléphoniques (voir la Figure\ref{fig:tech-adoption}). Les lignes
téléphoniques, l'électricité, les ordinateurs, Internet, les smartphones ; tous
observent des tendances exponentielles en termes de qualité-prix et d'adoption.
C'est pareil pour Bitcoin~\cite{tech-adoption}.

\begin{figure}
  \includegraphics{assets/images/tech-adoption.png}
  \caption{Bitcoin est littéralement hors-normes.}
  \label{fig:tech-adoption}
\end{figure}

Bitcoin possède de multiples effets de réseau\footnote{Trace Mayer, \textit{Les
sept effets de réseau de Bitcoin}~\cite{7-network-effects}} découlant tous de
configurations de croissance exponentielle dans leurs propres domaines : le
prix, les usagers, la sécurité, les développeurs, la part de marché et
l'adoption comme monnaie globale.

En ayant survécu à ses premiers pas, Bitcoin continue de grandir chaque jour
dans plus d'un aspect. D'accord, sa technologie n'est pas encore mature. Il est
sans doute en pleine adolescence. Mais si la technologie est exponentielle,
passer de l'ombre à l'omniprésence sera un court chemin.

\begin{figure}
  \includegraphics{assets/images/mobile-phone.png}
  \caption{Le téléphone mobile, env. 1965 contre 2019.}
  \label{fig:mobile-phone}
\end{figure}

Dans sa conférence TED de 2003, Jeff Bezos a choisi l'électricité comme
métaphore au futur du web\footnote{\url{http://bit.ly/bezos-web}}. Ces trois
phénomènes --- l'électricité, Internet, Bitcoin --- sont des technologies
\textit{habilitantes}, des réseaux qui facilitent autre chose. Ce sont des
infrastructures fondamentales, qui permettent de bâtir.

Ça fait un moment que nous côtoyons l'électricité. On la prend pour acquise.
Internet est un peu plus jeune mais la plupart des gens le prennent aussi pour
acquis. Bitcoin a dix ans et n'a pénétré les consciences que durant le dernier
cycle d'engouement. Seuls les pionniers le prennent pour acquis. Plus le temps
passera, plus les gens verront Bitcoin comme une chose banale\footnote{Ceci est
connu sous le nom d'\textit{effet Lindy}. L'effet Lindy est la théorie selon
laquelle l'espérance de vie future d'une chose non périssable est
proportionnelle à son âge actuel, impliquant une espérance de vie restante plus
longue à chaque fois qu'elle survit à une période de temps.~\cite{wiki:lindy}}. 

En 1994, Internet était encore déroutant et contre-intuitif. Il suffit de
regarder ce vieil enregistrement du \textit{Today
Show}\footnote{\url{https://youtu.be/UlJku_CSyNg}} pour voir qu'à l'évidence, ce
qui nous paraît naturel et intuitif aujourd'hui ne l'était en fait pas à
l'époque. Pour la plupart, Bitcoin reste encore déroutant et étrange, mais tout
comme Internet est une seconde nature pour les natifs du numérique, dépenser et
accumuler des sats\footnote{\url{https://twitter.com/hashtag/stackingsats}} le
sera pour les natifs du Bitcoin dans le futur.

\begin{quotation}\begin{samepage}
\enquote{Le futur est déjà là --- il n’est simplement pas réparti
équitablement.}
\begin{flushright} -- William Gibson\footnote{William Gibson, \textit{La science
dans la science-fiction} \cite{william-gibson}}
\end{flushright}\end{samepage}\end{quotation}

En 1995, environ $15\%$ des adultes américains utilisaient Internet. Les données
historiques du centre de recherches Pew~\cite{pew-research} montrent à quel
point Internet s'est immiscé dans nos vies. Selon un sondage client de Kaspersky
Lab~\cite{web:kaspersky}, $13\%$ des personnes interrogées se sont servi de
Bitcoin ou de ses clones pour payer un bien en 2018. Bien que les paiements ne
soient pas l'unique cas d'usage de Bitcoin, cela donne une idée d'où nous en
sommes en temps Internet : dans la première moitié des années 90.

En 1997, Jeff Bezos écrivait à ses actionnaires~\cite{bezos-letter}
\enquote{c'est le premier jour d'Internet}, pressentant son gigantesque
potentiel inexploité et, par extension, celui de son entreprise. Peu importe à
quel jour se trouve Bitcoin, seul l'observateur inattentif ne voit pas
clairement les immenses volumes de potentiel inexploité.

\begin{figure}
  \includegraphics{assets/images/internet-evolution-white-dates.png}
  \caption{Internet, 1982 vs. 2005. Source : CC-BY Merit Network, Inc. et
  Barrett Lyon, Opte Project}
  \label{fig:internet-evolution-white-dates}
\end{figure}

Le premier nœud Bitcoin fut mis en ligne en 2009 après que Satoshi mina le
\textit{bloc de genèse}\footnote{Le bloc de genèse est le premier bloc de la
chaîne de blocs Bitcoin. Les versions modernes de Bitcoin le numérotent $0$,
alors que les anciennes versions le comptent comme le bloc $1$. Le bloc de
genèse est habituellement codé en dur dans les applications qui se servent de la
chaîne de blocs de Bitcoin. C'est un cas particulier car il ne référence pas de
bloc précédent et produit une récompense non-dépensable. Le paramètre
\textit{coinbase} contient, entres autres données normales, le texte suivant :
\textit{\enquote{The Times 03/Jan/2009 Chancellor on brink of second bailout for
banks}} \cite{btcwiki:genesis-block}} et libéra le logiciel dans la nature. Son
nœud ne fut pas seul très longtemps. Hal Finney fut l'un des premiers à
accrocher à l'idée et à rejoindre le réseau. Dix ans plus tard, au moment où
j'écris ceci, il y a plus de $75 000$ nœuds qui exécutent bitcoin.

\begin{figure}
  \centering
  \includegraphics[width=8cm]{assets/images/running-bitcoin.png}
  \caption{Hal Finney est l'auteur du premier tweet à mentionner bitcoin en
  janvier 2009.}
  \label{fig:running-bitcoin}
\end{figure}

La couche de base du protocole n'est pas la seule à croître de façon
exponentielle. Le réseau Lightning, une technologie de seconde couche, grandit
encore plus vite.

En janvier 2018, le réseau Lightning était composé de $40$ nœuds et $60$
canaux~\cite{web:lightning-nodes}. En avril 2019, le réseau s'était étendu à
plus de $4000$ nœuds et environ $40 000$ canaux. N'oubliez pas que ça reste une
technologie expérimentale où la perte de fonds peut arriver et arrive parfois.
Malgré ça, la tendance est limpide : des milliers de personnes téméraires sont
enthousiastes à l'idée de s'en servir. 

\begin{figure}
  \includegraphics{assets/images/lnd-growth-lopp-white.png}
  \caption{Le réseau Lightning, janvier 2018 vs. décembre 2018. Source : Jameson
  Lopp}
  \label{fig:lnd-growth-lopp-white.png}
\end{figure}

À mon sens, ayant vécu l'essor météorique du web, les analogies entre Internet
et Bitcoin sont évidentes. Ce sont tous deux des réseaux, des technologies
exponentielles et tous deux amènent de nouvelles possibilités, de nouvelles
industries, de nouveaux comportements. L'électricité est la meilleure métaphore
pour comprendre la direction que prend Internet, il est donc possible
qu'Internet soit la meilleure métaphore pour comprendre la direction que prend
Bitcoin. Ou alors, pour reprendre les mots d'Andreas Antonopoulos, Bitcoin est
l'\textit{Internet de l'argent}. Ces métaphores sont un très bon rappel d'une
Histoire qui ne se répète pas mais qui rime souvent.

Les technologies exponentielles sont difficiles à appréhender et sont souvent
sous-estimées. Même si je m'intéresse beaucoup à celles-ci, je suis sans cesse
surpris de l'allure du progrès et de l'innovation. Observer la croissance de
l'écosystème Bitcoin, ça ressemble à observer l'essor d'Internet, mais en
accéléré. C'est grisant.

Ma quête de sens envers Bitcoin m'a mené plus d'une fois sur les chemins de
l'Histoire. La compréhension des anciennes structures sociétales, des monnaies
du passé et de comment les réseaux de communication ont évolué ont toutes fait
partie du voyage. Du biface au smartphone, la technologie a sans nul doute
changé notre monde à de nombreuses reprises. Les technologies de réseaux ont un
caractère particulier de transformation : l'écriture, les routes, l'électricité,
Internet. Toutes ont changé le monde. Bitcoin a changé le mien et continuera de
transformer les esprits et les cœurs de ceux qui osent l'approcher.

\paragraph{Bitcoin m'a appris que la compréhension du passé était nécessaire à
la compréhension de son futur. Un futur qui ne fait que commencer\ldots}

% ---
%
% #### Down the Rabbit Hole
%
% - [The Rising Speed of Technological Adoption][the rising speed of technological adoption] by Jeff Desjardins
% - [The 7 Network Effects of Bitcoin][multiple network effects] by Trace Mayer
% - [The Electricity Metaphor for the Web's Future][TED talk] by Jeff Bezos
% - [How the internet has woven itself into American life][data from the Pew Research Center] by Susannah Fox and Lee Rainie
% - [Genesis Block][genesis block] on the Bitcoin Wiki
% - [Lindy Effect][more time] on Wikipedia
%
% [Our World in Data]: https://ourworldindata.org/
% [the rising speed of technological adoption]: https://www.visualcapitalist.com/rising-speed-technological-adoption/
% [multiple network effects]: https://www.thrivenotes.com/the-7-network-effects-of-bitcoin/
% [TED talk]: https://www.ted.com/talks/jeff_bezos_on_the_next_web_innovation
% [recording of the Today Show]: https://www.youtube.com/watch?v=UlJku_CSyNg
% [William Gibson]: https://www.npr.org/2018/10/22/1067220/the-science-in-science-fiction
% [data from the Pew Research Center]: https://www.pewinternet.org/2014/02/27/part-1-how-the-internet-has-woven-itself-into-american-life/
% [consumer survey]: https://www.kaspersky.com/blog/money-report-2018/
% [letter to shareholders]: http://media.corporate-ir.net/media_files/irol/97/97664/reports/Shareholderletter97.pdf
% [running bitcoin]: https://twitter.com/halfin/status/1110302988?lang=en
% [40 nodes]: https://bitcoinist.com/bitcoin-lightning-network-mainnet-nodes/
% [reckless]: https://twitter.com/hashtag/reckless
% [Jameson Lopp]: https://twitter.com/lopp/status/1077200836072296449
% [\textit{The Internet of Money}]: https://theinternetofmoney.info/
% [stacking]: https://twitter.com/hashtag/stackingsats
%
% <!-- Bitcoin Wiki -->
% [genesis block]: https://en.bitcoin.it/wiki/Genesis_block
%
% <!-- Wikipedia -->
% [more time]: https://en.wikipedia.org/wiki/Lindy_effect
% [alice]: https://en.wikipedia.org/wiki/Alice%27s_Adventures_in_Wonderland
% [carroll]: https://en.wikipedia.org/wiki/Lewis_Carroll
