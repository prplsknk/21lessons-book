\chapter{Les cypherpunks écrivent du code}
\label{les:20}

\begin{chapquote}{Lewis Carroll, \textit{Alice au pays des merveilles}}
\enquote{Je vois bien que vous essayez d’inventer quelque chose !}
\end{chapquote}

Comme beaucoup de grandes idées, Bitcoin n'est pas sorti de nulle part. Il a vu
le jour en utilisant et en combinant beaucoup d'innovations et de découvertes en
mathématiques, en physique, en informatique et dans d'autres domaines. Satoshi
est sans conteste un génie mais il n'aurait pas pu inventer Bitcoin sans se
tenir sur des épaules de géants.

\begin{quotation}\begin{samepage}

\enquote{Celui qui simplement souhaite et espère n'intervient pas activement
dans le cours des événements ni dans le profil de sa destinée.}
\begin{flushright} -- Ludwig von Mises\footnote{Ludwig von Mises,
\textit{L’Action Humaine} \cite{human-action}}
\end{flushright}\end{samepage}\end{quotation}
% > <cite>[Ludwig Von Mises]</cite>

L'un de ces géants, c'est Eric Hughes, un des fondateurs du mouvement cypherpunk
et auteur du \textit{Manifeste d'un Cypherpunk}. Ça paraît difficile d'imaginer
que Satoshi n'ait pas été influencé par ce manifeste. Il parle de tellement de
choses que Bitcoin permet et utilise, telles que les transactions privées
directes, l'argent électronique et liquide, les systèmes anonymes et la
protection de la vie privée par la cryptographie et les signatures numériques.

\begin{quotation}\begin{samepage}
\enquote{La confidentialité est nécessaire pour une société ouverte à l'ère
électronique. [...] Puisque nous désirons la confidentialité, nous devons nous
assurer que chaque partie à une transaction n'a connaissance que de ce qui est
directement nécessaire à cette transaction. [...]
Par conséquent, la vie privée dans une société ouverte nécessite des systèmes de
transaction anonymes. Jusqu'à présent, l'argent liquide a été le principal
système de ce type. Un système de transaction anonyme n'est pas un système de
transaction secret. [...]
Nous les Cypherpunks sommes dédiés à la construction de systèmes anonymes. Nous
défendons notre vie privée avec la cryptographie, avec des systèmes de transfert
de courrier anonyme, avec des signatures numériques et avec de la monnaie
électronique.
Les Cypherpunks écrivent du code.}
\begin{flushright} -- Eric Hughes\footnote{Eric Hughes, Manifeste d'un
Cypherpunk \cite{cypherpunk-manifesto}}
\end{flushright}\end{samepage}\end{quotation}

Les Cypherpunks ne trouvent pas de réconfort dans les espoirs et les vœux. Ils
s'immiscent activement dans le cours des événements et forgent leur propre
destinée. Les Cypherpunks écrivent du code.

Et donc, en fidèle cypherpunk, Satoshi s'est assis et s'est mis à coder. Du code
parti d'une idée abstraite pour prouver au monde qu'elle pouvait marcher. Du
code semant la graine d'une nouvelle réalité économique. Grâce au code, chacun
peut vérifier que ce système novateur fonctionne vraiment et qu'à peu près
toutes les 10 minutes, Bitcoin prouve au monde qu'il est encore vivant.

\begin{figure}
  \includegraphics{assets/images/bitcoin-code-white.png}
  \caption{Extraits du code de la version 0.1 de Bitcoin}
  \label{fig:bitcoin-code-white}
\end{figure}

Afin de s'assurer que son invention ne resterait pas du domaine du rêve, Satoshi
a écrit le code de son idée avant d'écrire le livre blanc. Il a aussi pris soin
de ne pas retarder\footnote{\enquote{Nous ne devrions pas reporter indéfiniment
tant que chaque fonction n'est pas terminée.} -- Satoshi
Nakamoto~\cite{satoshi-delay}} chaque version indéfiniment. Après tout,
\enquote{il y aura toujours autre chose à faire}.

\begin{quotation}\begin{samepage}
\enquote{J'ai dû écrire tout le code avant d'être convaincu que je pouvais
résoudre chaque problème, puis j'ai écrit le livre blanc.}
\begin{flushright} -- Satoshi Nakamoto\footnote{Satoshi Nakamoto, dans Re:
Bitcoin P2P e-cash paper \cite{satoshi-mail-code-first}}
\end{flushright}\end{samepage}\end{quotation}

Dans ce monde aux promesses infinies et au déroulement douteux, la mise en
pratique d'un développement dévoué manquait cruellement. Soyez volontaires,
persuadez-vous d'être capable de résoudre les problèmes et implémentez les
solutions. On devrait tous essayer d'être un peu plus cypherpunk.

\paragraph{Bitcoin m'a appris que les cypherpunks écrivaient du code.}

% ---
%
% #### Down the Rabbit Hole
%
% - [Bitcoin version 0.1.0 announcement][version 0.1.0] by Satoshi Nakamoto
% - [Bitcoin P2P e-cash paper announcement][mail-announcement] by Satoshi Nakamoto
%
% [mail-announcement]: http://www.metzdowd.com/pipermail/cryptography/2008-October/014810.html
% [Ludwig Von Mises]: https://mises.org/library/human-action-0/html/pp/613
% [version 0.1.0]: https://bitcointalk.org/index.php?topic=68121.0
% [not to delay]: https://bitcointalk.org/index.php?topic=199.msg1670#msg1670
% [6]: http://www.metzdowd.com/pipermail/cryptography/2008-November/014832.html
%
% <!-- Wikipedia -->
% [alice]: https://en.wikipedia.org/wiki/Alice%27s_Adventures_in_Wonderland
% [carroll]: https://en.wikipedia.org/wiki/Lewis_Carroll
