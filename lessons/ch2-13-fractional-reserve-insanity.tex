\chapter{La folie des réserves fractionnaires}
\label{les:13}

\begin{chapquote}{Lewis Carroll, \textit{Alice au pays des merveilles}}
Hélas ! les regrets étaient inutiles ! Elle continuait à grandir sans arrêt, et,
bientôt, elle fût obligée de s’agenouiller sur le plancher : une minute plus
tard, elle n’avait même plus assez de place pour rester à genoux, et elle
essayait de voir si elle serait mieux en se couchant, un coude contre la porte,
son autre bras replié sur la tête. Puis, comme elle ne cessait toujours pas de
grandir, elle passa un bras par la fenêtre, mit un pied dans la cheminée, et se
dit : \enquote{À présent je ne peux pas faire plus, quoi qu’il arrive. Que
vais-je devenir ?}
\end{chapquote}

La valeur et la monnaie ne sont pas des sujets simples, particulièrement de nos
jours. Le processus d'émission monétaire du système bancaire ne l'est pas plus
et je ne peux m'empêcher de croire que c'est délibéré. Ce phénomène, que je
n'avais jusqu'à présent rencontré que dans les papiers de recherche et les
documents légaux, semble également courant dans les milieux financiers : rien
n'est expliqué de façon simple, non pas parce que c'est réellement complexe,
mais parce que la vérité est dissimulée sous des couches et des couches de
jargon d'une complexité \textit{apparente}. \enquote{Politique monétaire
expansionniste, assouplissement quantitatif, relance fiscale de l'économie}. Le
public hoche de la tête, hypnotisé par les mots savants.

Les banques à réserve fractionnaire et l'assouplissement quantitatif sont
justement deux de ces mots sophistiqués, dissimulant la réalité des choses en la
présentant comme complexe et difficile à comprendre. Si vous deviez expliquer ça
à un enfant, vous verriez rapidement la folie qu'ils renferment.

Godfrey Bloom l'a exprimé bien mieux que je ne pourrais jamais le faire, en
s'adressant au Parlement Européen lors d'un débat commun :

\begin{quotation}\begin{samepage}
\enquote{[...] vous ne comprenez pas vraiment le concept de banque. Toutes les
banques sont fauchées. La Santander, la Deutsche Bank, la Royal Bank of Scotland
--- elles sont toutes fauchées ! Et pourquoi elles sont fauchées ? Ce
n'est pas une volonté divine. Ce n'est pas une espèce de tsunami. Elles sont
fauchées parce qu'elles ont un système appelé `réserves fractionnaires' qui leur
permet de prêter de l'argent qu'elles n'ont même pas ! C'est un scandale
orchestré par des malfaiteurs et ça fait trop longtemps que ça dure. [...] Il y
a de la contrefaçon --- qu'on appelle parfois assouplissement quantitatif ---
mais ça reste de la contrefaçon sous un autre nom. La création monétaire
artificielle coûterait au simple quidam un long moment derrière les barreaux
[...] et tant que l'on n'enverra pas les banquiers --- et j'inclus ici les
banquiers centraux et les politiques --- en prison pour ce scandale cela
continuera.}
\begin{flushright} -- Godfrey Bloom\footnote{Débat commun sur l'union
bancaire~\cite{godfrey-bloom}}
\end{flushright}\end{samepage}\end{quotation}

Permettez-moi de répéter le passage essentiel : les banques peuvent prêter de
l'argent qu'elles ne possèdent pas.

Grâce au système de réserve fractionnaire, une banque a besoin de ne garder
qu'une \textit{fraction} de chaque dollar qu'elle reçoit. Ça se situe quelque
part entre $0$ et $10\%$, plutôt vers la limite inférieure d'ailleurs, ce qui
n'arrange rien.

Prenons un exemple concret afin de mieux illustrer cette idée insensée : une
fraction de $10\%$ fera l'affaire et nous devrions pouvoir calculer de tête. 
Avantageux pour tout le monde. Donc, si vous déposez 100\$ à la banque ---
parce que vous n'avez pas envie de les garder sous votre matelas --- celle-ci
n'a besoin de garder que la \textit{fraction} consentie. Dans notre exemple,
cela fait 10\$, car 10\% de 100\$ font 10\$. Simple, non ?

Mais alors que font les banques avec le reste de l'argent ? Qu'arrive-t-il aux
90\$ restants ? Elles font ce que les banques savent faire, elles les prêtent à
d'autres. Cela produit un effet multiplicateur sur la monnaie, qui accroît
énormément son offre dans l'économie (Figure~\ref{fig:money-multiplier}). Votre
dépôt initial de 100\$ se transforme rapidement en 190\$. En prêtant 90\% de ces
90\$ fraîchement créés, cela fera bientôt 271\$ dans l'économie. Et 343,90\$
après ça. L'offre de monnaie gonfle de façon récursive, puisque les banques
prêtent littéralement de l'argent qu'elles n'ont
pas~\cite{wiki:money-multiplier}. Sans la moindre formule, les banques changent
100\$ en plus d'un milier, comme par magie. Il s'avère qu'il est simple
d'arriver à un facteur 10. Ça prend seulement quelques cycles de prêt.

\begin{figure}
  \centering
  \includegraphics{assets/images/money-multiplier.png}
  \caption{L'effet multiplicateur sur la monnaie}
  \label{fig:money-multiplier}
\end{figure}

\paragraph{}
Ne vous méprenez pas : il n'y a rien de mal à prêter. Il n'y a rien de mal à
percevoir des intérêts. Il n'y a même rien de mal avec cette bonne vieille
banque qui garde votre patrimoine plus sûrement que dans votre tiroir à
chaussettes.

Les banques centrales sont une toute autre affaire, en revanche. Elles sont les
abominations de la régulation financière, mi-publiques mi-privées, jouant à Dieu
avec des choses qui impactent tout membre de la civilisation mondiale, sans
morale, avec pour seul intérêt le futur à court terme et apparemment aucune
responsabilité ni vérifiabilité (voir la Figure~\ref{fig:bsg}).

\begin{figure}
  \centering
  \includegraphics{assets/images/bsg.jpg}
  \caption{Yellen est fermement opposée à un audit de la réserve fédérale,
  pendant que le gars au panneau Bitcoin soutient fermement l'achat de bitcoin.}
  \label{fig:bsg}
\end{figure}

Bien que Bitcoin soit encore inflationniste, il cessera de l'être relativement
rapidement. La limite stricte sur l'offre de 21 millions de bitcoins finira par
éliminer totalement l'inflation. Nous avons maintenant deux mondes monétaires :
l'un inflationniste où l'argent est créé arbitrairement et le monde de Bitcoin,
où l'offre finale est fixe et facilement vérifiable par tout un chacun. L'un
nous est imposé par la violence, l'autre peut être rejoint par quiconque le
désire. Pas de barrière à l'entrée, personne à qui demander la permission. Une
participation volontaire. Voilà la beauté de Bitcoin.

J'ajouterais que le débat entre les économistes Keynésiens\footnote{Les théories
de John Maynard Keynes et ses adeptes~\cite{wiki:keynesian}} et l'école
autrichienne\footnote{École de pensée économique basée sur l'individualisme
méthodologique~\cite{wiki:austrian}} n'est plus seulement académique. Satoshi
est parvenu à créer un système de transfert de valeur sous stéroïdes, inventant
dans le même temps la monnaie la plus saine ayant jamais existé. D'une façon ou
d'une autre, de plus en plus de gens prendront conscience de l'arnaque qu'est le
système de réserve fractionnaire. S'ils en arrivent aux mêmes conclusions que la
plupart des Bitcoiners et économistes de l'école autrichienne, ils pourraient
rejoindre l'internet de l'argent, qui ne cesse de grandir. Personne ne peut les
empêcher de faire ce choix.

\paragraph{Bitcoin m'a appris que les réserves fractionnaires des banques
n'étaient que pure folie.}

% ---
%
% #### Down the Rabbit Hole
%
% - [The Creature From Jekyll Island] by G. Edward Griffin
% - [Money Multiplier][money multiplier], [Keynesian Economics][Keynesian], [Austrian School][Austrian] on Wikipedia
%
% [The Creature From Jekyll Island]: https://archive.org/details/pdfy--Pori1NL6fKm2SnY
%
% [joint debate]: https://www.youtube.com/watch?v=hYzX3YZoMrs
% [money multiplier]: https://en.wikipedia.org/wiki/Money_multiplier
% [auditability]: https://i.ytimg.com/vi/ThFGs347MW8/maxresdefault.jpg
% [Keynesian]: https://en.wikipedia.org/wiki/Keynesian_economics
% [Austrian]: https://en.wikipedia.org/wiki/Austrian_School
%
% <!-- Wikipedia -->
% [alice]: https://en.wikipedia.org/wiki/Alice%27s_Adventures_in_Wonderland
% [carroll]: https://en.wikipedia.org/wiki/Lewis_Carroll
